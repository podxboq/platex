\chapter{Encoding and Decoding Stabilizer Codes}
\label{chap-encoding}

\section{Standard Form for a Stabilizer Code}

To see how to encode a general stabilizer code~\cite{cleve}, it is helpful to
describe the code in the language of binary vector spaces (see section
\ref{sec-alternate}).  Note that the specific choice of generators is not
at all unique.  We can always replace a generator $M_i$ with $M_i M_j$ for
some other generator $M_j$.  The corresponding effect on the binary matrices
is to add row $j$ to row $i$ in both matrices.  For simplicity, it is also
helpful to rearrange qubits in the code.  This has the effect of rearranging
the corresponding columns in both matrices.  Combining these two operations,
we can perform Gaussian elimination on the first matrix, putting the code in
this form:
\vspace{2.5ex}
\begin{equation}
	\begin{array}{r} r\{ \\ n-k-r\{ \end{array} \!\!\!\!
	\left( \begin{array}{cc|cc}
		       \raisebox{0ex}[1.5ex]{$\overbrace{I}^r$} &
		       \raisebox{0ex}[1.5ex]{$\overbrace{A}^{n-r}$} &
		       \raisebox{0ex}[1.5ex]{$\overbrace{B}^r$} &
		       \raisebox{0ex}[1.5ex]{$\overbrace{C}^{n-r}$} \\
		       0 & 0 & D & E
	\end{array} \right).
\end{equation}
Here, $r$ is the rank of the $\X$ portion of the stabilizer generator matrix.

Then we perform another Gaussian elimination on $E$ to get
\vspace{2.5ex}
\begin{equation}
	\begin{array}{r} r\{ \\ n-k-r-s \{ \\ s \{ \end{array} \!\!\!\!
	\left( \begin{array}{ccc|ccc}
		       \raisebox{0ex}[1.5ex]{$\overbrace{I}^r$} &
		       \raisebox{0ex}[1.5ex]{$\overbrace{A_1}^{n-k-r-s}$} &
		       \raisebox{0ex}[1.5ex]{$\overbrace{A_2}^{k+s}$} &
		       \raisebox{0ex}[1.5ex]{$\overbrace{B}^r$} &
		       \raisebox{0ex}[1.5ex]{$\overbrace{C_1}^{n-k-r-s}$} &
		       \raisebox{0ex}[1.5ex]{$\overbrace{C_2}^{k+s}$} \\
		       0 & 0 & 0 & D_1 & I & E_2 \\
		       0 & 0 & 0 & D_2 & 0 & 0
	\end{array} \right).
\end{equation}
The rank of $E$ is $n-k-r-s$.  However, the first $r$ generators will not
commute with the last $s$ generators unless $D_2 = 0$, which really implies
that $s=0$.  Thus we can always put the code into the standard form
\vspace{2.5ex}
\begin{equation}
	\begin{array}{r} r\{ \\ n-k-r\{ \end{array} \!\!\!\!
	\left( \begin{array}{ccc|ccc}
		       \raisebox{0ex}[1.5ex]{$\overbrace{I}^r$} &
		       \raisebox{0ex}[1.5ex]{$\overbrace{A_1}^{n-k-r}$} &
		       \raisebox{0ex}[1.5ex]{$\overbrace{A_2}^k$} &
		       \raisebox{0ex}[1.5ex]{$\overbrace{B}^r$} &
		       \raisebox{0ex}[1.5ex]{$\overbrace{C_1}^{n-k-r}$} &
		       \raisebox{0ex}[1.5ex]{$\overbrace{C_2}^k$} \\
		       0 & 0 & 0 & D & I & E
	\end{array} \right).
	\label{eq-standard-form}
\end{equation}
For instance, the standard form for the five-qubit code of
table~\ref{table-5qubit} is
\begin{equation}
	\left( \begin{array}{ccccc|ccccc}
		       1 & 0 & 0 & 0 & 1 & 1 & 1 & 0 & 1 & 1 \\
		       0 & 1 & 0 & 0 & 1 & 0 & 0 & 1 & 1 & 0 \\
		       0 & 0 & 1 & 0 & 1 & 1 & 1 & 0 & 0 & 0 \\
		       0 & 0 & 0 & 1 & 1 & 1 & 0 & 1 & 1 & 1
	\end{array} \right).
\end{equation}

Suppose we have an $\Xbar$ operator which in this language is written
$(u|v) = (u_1 u_2 u_3| v_1 v_2 v_3)$, where $u_1$ and $v_1$ are
$r$-dimensional vectors, $u_2$ and $v_2$ are $(n-k-r)$-dimensional
vectors, and $u_3$ and $v_3$ are $k$-dimensional vectors.  However,
elements of $N(S)$ are equivalent up to multiplication by elements of $S$.
Therefore, we can also perform eliminations on $\Xbar$ to force $u_1 = 0$
and $v_2 = 0$.  Then, because $\Xbar$ is in $N(S)$, we must satisfy
(\ref{eq-commute-bin}), so
\begin{eqnarray}
	\left( \begin{array}{cccccc}
		       I & A_1 & A_2 & B & C_1 & C_2 \\
		       0 & 0 & 0 & D & I & E
	\end{array} \right)
	\left( \begin{array}{c} v_1^T \\ 0 \\ v_3^T \\ 0 \\ u_2^T \\ u_3^T
	\end{array} \right)
	& \!\!\! = & \!\!\! \left( \begin{array}{c}
		                           v_1^T + A_2 v_3^T + C_1 u_2^T + C_2 u_3^T \\
		                           u_2^T + E u_3^T
	\end{array} \right) \nonumber \\
	& \!\!\! = & \!\!\! \left( \begin{array}{c} 0 \\ 0 \end{array} \right).
	\label{eq-Xbar-commute}
\end{eqnarray}

Suppose we want to choose a complete set of $k$ $\Xbar$ operators.  We
can combine their vectors into two $k \times n$ matrices $(0 U_2 U_3 |
V_1 0 V_3)$.  We want them to commute with each other, so $U_3 V_3^T +
V_3 U_3^T = 0$.  Suppose we pick $U_3 = I$.  Then we can take $V_3 = 0$, and
by equation (\ref{eq-Xbar-commute}), $U_2 = E^T$ and $V_1 = E^T C_1^T + C_2^T$.
The rest of the construction will assume that this choice has actually been
made.  Another choice of $U_3$ and $V_3$ will require us to perform some
operation on the unencoded data to compensate.  For the five-qubit code, the
standard form of the $\Xbar$ generator would be $(0 0 0 0 1 | 1 0 0 1 0)$.
We can see that this is equivalent (mod $S$) to the $\Xbar$ given in
table~\ref{table-5qubit}.

We can also pick a complete set of $k$ $\Zbar$ operators, which act on the
code as encoded $\Z$ operators.  They are uniquely defined (up to
multiplication by $S$, as usual) given the $\Xbar$ operators.  $\Zbar_i$ is
an operator that commutes with $M \in S$, commutes with $\Xbar_j$ for $i
\neq j$, and anticommutes with $\Xbar_i$.  We can bring it into the
standard form $(0 U_2' U_3' | V_1' 0 V_3')$.  Then
\begin{equation}
	U_3' V_3^T + V_3' U_3^T = I.
\end{equation}
When $U_3 = I$ and $V_3 = 0$, $V_3' = I$.  Since equation
(\ref{eq-Xbar-commute}) holds for the $\Zbar$ operators too, $U_2' = U_3'
= 0$ and $V_1' = A_2^T$.  For instance, for the five-qubit code, the standard
form of the $\Zbar$ generator is $(0 0 0 0 0 | 1 1 1 1 1)$, which is exactly
what is given in table~\ref{table-5qubit}.

\section{Network for Encoding}
\label{sec-encode-network}

Given a stabilizer in standard form along with the $\Xbar$ operators in
standard form, it is straightforward to produce a network to encode the
corresponding code.  The operation of encoding a stabilizer code can be
written as
\begin{eqnarray}
	\ket{c_1 \ldots c_k} & \rightarrow & \left( \Sum_{M \in S} M \right)
	\Xbar_1^{c_1} \cdots \Xbar_k^{c_k} \ket{0 \ldots 0} \label{eq-encoding}\\
	& = & (I + M_1) \cdots (I + M_{n-k}) \Xbar_1^{c_1} \cdots \Xbar_k^{c_k}
	\ket{0\ldots 0}, \label{eq-encoding2}
\end{eqnarray}
where $M_1$ through $M_{n-k}$ generate the stabilizer, and $\Xbar_1$
through $\Xbar_k$ are the encoded $\X$ operators for the $k$ encoded
qubits.  This is true because, in general, for any $N \in S$,
\begin{equation}
	N \left( \Sum_{M \in S} M \right) \ket{\psi} = \left( \Sum_{M \in S} NM
	\right) \ket{\psi} = \left( \Sum_{M' \in S} M' \right) \ket{\psi},
\end{equation}
so $\Sum M \ket{\psi}$ is in the coding space $T$ for any state
$\ket{\psi}$.  If we define the encoded $0$ as
\begin{equation}
	\ket{\overline{0}} = \Sum_{M \in S} M \ket{\overbrace{0 \ldots 0}^{n}},
\end{equation}
then by the definition of the $\Xbar$'s, we should encode
\begin{equation}
	\ket{c_1 \ldots c_k} \rightarrow \Xbar_1^{c_1} \cdots \Xbar_k^{c_k} \left(
	\Sum_{M \in S} M \right)  \ket{0 \ldots 0}.
\end{equation}
Since $\Xbar_i$ commutes with $M \in S$, this is just (\ref{eq-encoding}).
Naturally, to encode this, we only need to worry about encoding the basis
states $\ket{c_1 \ldots c_k}$.

The standard form of $\Xbar_i$ has the form $Z^{(r)} X^{(n-k-r)}
\Xs{(n-k+i)}$ ($Z^{(r)}$ is the product of $\Z$'s on the first $r$ qubits and
$X^{(n-k-r)}$ is the product of $\X$'s on the next $n-k-r$ qubits).  Suppose
we put the $k$th input qubit $\ket{c_k}$ in the $n$th spot,
following $n-1$ $0$s.  The state $\Xbar_k^{c_k} \ket{0 \ldots 0}$ therefore
has a $1$ for the $n$th qubit iff $\ket{c_k} = \ket{1}$.  This means we can
get the state $\Xbar_k^{c_k} \ket{0 \ldots 0}$ by applying $\Xbar_k$
(without the final $\Xs{n}$) to the input state conditioned on the $n$th
qubit.  For instance, for the five-qubit code, $\Xbar = Z \otimes I \otimes I
\otimes Z \otimes X$.  The corresponding operation is illustrated in
figure~\ref{fig-5qubit-Xbar}.
\begin{figure}
	\centering
	\begin{picture}(60,120)

		\put(20,20){\line(1,0){40}}
		\put(20,40){\line(1,0){14}}
		\put(46,40){\line(1,0){14}}
		\put(20,60){\line(1,0){40}}
		\put(20,80){\line(1,0){40}}
		\put(20,100){\line(1,0){14}}
		\put(46,100){\line(1,0){14}}

		\put(2,14){\makebox(12,12){$c$}}
		\put(2,34){\makebox(12,12){$0$}}
		\put(2,54){\makebox(12,12){$0$}}
		\put(2,74){\makebox(12,12){$0$}}
		\put(2,94){\makebox(12,12){$0$}}

		\put(40,20){\circle*{4}}
		\put(40,20){\line(0,1){14}}
		\put(34,34){\framebox(12,12){$\Z$}}
		\put(40,46){\line(0,1){48}}
		\put(34,94){\framebox(12,12){$\Z$}}

	\end{picture}
	\caption{Creating the state $\Xbar \ket{00000}$ for the five-qubit code.}
	\label{fig-5qubit-Xbar}
\end{figure}
In this case $r=n-k=4$, so there are no bit flips, only controlled $\Z$'s.

In the more general case, we also need to apply $\Xbar_1$ through
$\Xbar_{k-1}$, depending on $c_1$ through $c_{k-1}$.  Since the form of
the $\Xbar$'s ensures that each only operates on a single one of the last
$k$ qubits, we can substitute $\ket{c_i}$ for the $(n-k+i)$th qubit and
apply $\Xbar_i$ conditioned on it, as with $\ket{c_k}$.  This produces the
state $\Xbar_1^{c_1} \cdots \Xbar_k^{c_k} \ket{0 \ldots 0}$.

Further, note that the $\Xbar$ operators only act as $\Z$ on the first $r$
qubits and as $\X$ on the next $n-k-r$ qubits.  Since $\Z$ acts trivially on
$\ket{0}$, we can just ignore that part of the $\Xbar$'s when implementing
this part of the encoder, leaving just the controlled NOTs.  The first $r$
qubits automatically remain in the state $\ket{0}$ after this step of
encoding.  This means that for the five-qubit code, this step of encoding is
actually trivial, with no operations.  In general, this step is only necessary
if $r<n-k$.

For the next step of the encoding, we note that the standard form of the
first $r$ generators only applies a single bit flip in the first $r$ qubits.
This means that when we apply $I + M_i$, the resulting state will be the
sum of a state with $\ket{0}$ for the $i$th qubit and a state with $\ket{1}$
for the $i$th qubit.  We therefore apply the Hadamard transform
\begin{equation}
	R = \frac{1}{\sqrt{2}} \pmqty{1 & \ 1 \\ 1 & -1}
	\label{eq-hadamard}
\end{equation}
to the first $r$ qubits, putting each in the state $\ket{0} + \ket{1}$.  Then
we apply $M_i$ (for $i = 1, \ldots, r$) conditioned on qubit $i$ (ignoring
the factor of $\Xs{i}$).  While these operators may perform phase
operations on the first $r$ qubits, they do not flip them, so there is no risk
of one operation confusing the performance of another one.  The one
possible complication is when $M_i$ has a factor of $\Zs{i}$.  In this case,
$\Zs{i}$ only introduces a minus sign if the qubit is $\ket{1}$ anyway, so we
do not need to condition it on anything.  Just performing $\Zs{i}$ after the
Hadamard transform is sufficient.  For the five-qubit code, the full network
for encoding is given in figure~\ref{fig-5qubit}.
\begin{figure}
	\centering
	\begin{picture}(160,120)

		\put(20,20){\line(1,0){54}}
		\put(86,20){\line(1,0){48}}
		\put(146,20){\line(1,0){14}}

		\put(20,40){\line(1,0){14}}
		\put(46,40){\line(1,0){8}}
		\put(66,40){\line(1,0){8}}
		\put(86,40){\line(1,0){8}}
		\put(106,40){\line(1,0){54}}

		\put(20,60){\line(1,0){14}}
		\put(46,60){\line(1,0){48}}
		\put(106,60){\line(1,0){28}}
		\put(146,60){\line(1,0){14}}

		\put(20,80){\line(1,0){14}}
		\put(46,80){\line(1,0){28}}
		\put(86,80){\line(1,0){28}}
		\put(126,80){\line(1,0){34}}

		\put(20,100){\line(1,0){14}}
		\put(46,100){\line(1,0){8}}
		\put(66,100){\line(1,0){48}}
		\put(126,100){\line(1,0){8}}
		\put(146,100){\line(1,0){14}}

		\put(4,14){\makebox(12,12){$c$}}
		\put(4,34){\makebox(12,12){$0$}}
		\put(4,54){\makebox(12,12){$0$}}
		\put(4,74){\makebox(12,12){$0$}}
		\put(4,94){\makebox(12,12){$0$}}

		\put(34,34){\framebox(12,12){$R$}}
		\put(34,54){\framebox(12,12){$R$}}
		\put(34,74){\framebox(12,12){$R$}}
		\put(34,94){\framebox(12,12){$R$}}

		\put(54,34){\framebox(12,12){$\Z$}}
		\put(54,94){\framebox(12,12){$\Z$}}

		\put(80,100){\circle*{4}}
		\put(80,100){\line(0,-1){14}}
		\put(74,74){\framebox(12,12){$\Z$}}
		\put(80,74){\line(0,-1){28}}
		\put(74,34){\framebox(12,12){$\Z$}}
		\put(80,34){\line(0,-1){8}}
		\put(74,14){\framebox(12,12){$\Y$}}

		\put(100,80){\circle*{4}}
		\put(100,80){\line(0,-1){14}}
		\put(94,54){\framebox(12,12){$\Z$}}
		\put(100,54){\line(0,-1){8}}
		\put(94,34){\framebox(12,12){$\Z$}}
		\put(100,34){\line(0,-1){18}}
		\put(100,20){\circle{8}}

		\put(120,60){\circle*{4}}
		\put(120,60){\line(0,1){14}}
		\put(114,74){\framebox(12,12){$\Z$}}
		\put(120,86){\line(0,1){8}}
		\put(114,94){\framebox(12,12){$\Z$}}
		\put(120,60){\line(0,-1){44}}
		\put(120,20){\circle{8}}

		\put(140,40){\circle*{4}}
		\put(140,40){\line(0,1){14}}
		\put(134,54){\framebox(12,12){$\Z$}}
		\put(140,66){\line(0,1){28}}
		\put(134,94){\framebox(12,12){$\Z$}}
		\put(140,40){\line(0,-1){14}}
		\put(134,14){\framebox(12,12){$\Y$}}

	\end{picture}
	\caption{Network for encoding the five-qubit code.}
	\label{fig-5qubit}
\end{figure}

For more general codes, $r<n-k$, and there are $n-k-r$ generators that are
formed just of the tensor product of $\Z$'s.  However, we do not need to
consider such generators to encode.  Let $M$ be such a generator.  Since
$M$ commutes with all the other generators and every $\Xbar$, we can
commute $I + M$ through until it acts directly on $\ket{0 \ldots 0}$.
However, $\Z$ acts trivially on $\ket{0}$, so $I+M$ fixes $\ket{0 \ldots 0}$,
and in equation~(\ref{eq-encoding2}), we can skip any $M_i$ that is the tensor
product of $\Z$'s.  The effect of these operators is seen just in
the form of the $\Xbar$ operators, which must commute with them.

Applying each of the $\Xbar$ operators requires up to $n-k-r$ two-qubit
operations.  Each of the first $r$ qubits must be prepared with a Hadamard
transform and possibly a $\Z$, which we can combine with the Hadamard
transform.  Then applying each of the first $r$ generators requires up to
$n-1$ two-qubit operations.  The whole encoder therefore requires up to $r$
one-qubit operations and at most
\begin{equation}
	k(n-k-r) + r(n-1) \leq (k+r)(n-k) \leq n(n-k)
\end{equation}
two-qubit operations.

\section{Other Methods of Encoding and Decoding}

We can decode a code by performing the above network in reverse.  In
order to do this, we should first perform an error correction cycle, since
the network will not necessarily work properly on an encoded state.  Note
that in principle we can build a decoder that corrects while decoding.  We
can form a basis for the Hilbert space from the states $A \ket{\psi_i}$,
where $A \in \G$ and $\ket{\psi_i}$ is a basis state for the coding space
$T$.  The combined corrector/decoder would map $A \ket{\psi_i}$ to
$\ket{i} \otimes \ket{f(A)}$, where $f(A)$ is the error syndrome for $A$.
If $A$ is not a correctable error, $\ket{i}$ will not necessarily be the state
encoded by $\ket{\psi_i}$, but if $A$ is correctable, it will be.  It is not
usually worthwhile using a quantum network that does this, since the error
correction process is usually dealt with more easily using classical
measurements.  However, some proposed implementations of quantum computation
cannot be used to measure a single system~\cite{gershenfeld}, so this sort of
network would be necessary.  The decoding method presented in
\cite{divincenzo-decoder} can easily be adapted to produce networks that
simultaneously correct and decode.

One good reason not to decode by running the encoder backwards is that
most of the work in the encoder went into producing the encoded $0$.
There is no actual information in that state, so we might be able to save
time decoding if we could remove the information without dealing with the
structure of the encoded $0$.  We can do this by using the $\Xbar$ and
$\Zbar$ operators.  If we want to measure the $i$th encoded qubit
without decoding, we can do this by measuring the eigenvalue of
$\Zbar_i$.  If the eigenvalue is $+1$, the $i$th encoded qubit is $\ket{0}$;
if it is $-1$, the $i$th encoded qubit is $\ket{1}$.  In standard form,
$\Zbar_i$ is the tensor product of $\Z$'s.  That means it will have
eigenvalue $(-1)^P$, where $P$ is the parity of the qubits acted on by
$\Zbar_i$.  Therefore, if we apply a controlled-NOT from each of these
qubits to an ancilla qubit, we have performed a controlled-NOT from the
$i$th encoded qubit to the ancilla --- we will flip the ancilla iff the $i$th
encoded qubit is $\ket{1}$.

If the original state of the code is $\ket{\overline{0}} \ket{\psi} +
\ket{\overline{1}} \ket{\phi}$ (with the first ket representing the $i$th
	{\em logical} qubit) and the ancilla begins in the state $\ket{0}$, after
applying this CNOT operation, we have
\begin{equation}
	\ket{\overline{0}} \ket{\psi} \ket{0} + \ket{\overline{1}} \ket{\phi}
	\ket{1}.
\end{equation}
Now we apply $\Xbar_i$ conditioned on the ancilla qubit.  This will flip the
$i$th encoded qubit iff the ancilla is $\ket{1}$.  This produces the state
\begin{equation}
	\ket{\overline{0}} \ket{\psi} \ket{0} + \ket{\overline{0}} \ket{\phi}
	\ket{1} = \ket{\overline{0}} \left(\ket{\psi} \ket{0} + \ket{\phi} \ket{1}
	\right).
\end{equation}
The $i$th encoded qubit has been set to $0$ and the ancilla holds the state
that the $i$th encoded qubit used to hold.  The rest of the code has been
left undisturbed.  We can repeat this operation with each of the encoded
qubits, transferring them to $k$ ancilla qubits.  Each such operation requires
at most $2(n-k+1)$ two-qubit operations (since $\Zbar$ requires at most $r + 1$
operations and $\Xbar$ could require $n-k+1$ operations).  Therefore, the
full decoder uses at most $2k(n-k+1)$ operations, which is often less than
is required to encode.  At the end of the decoding, the original $n$ qubits
holding the code are left in the encoded $0$ state.

We can run this process backwards to encode, but we need an encoded $0$
state to begin with.  This could be a residue from an earlier decoding
operation, or could be produced separately.  One way to produce it would
be to use the network of section~\ref{sec-encode-network}, using
$\ket{0\ldots 0}$ as the input data.  Alternately, we could produce it by
performing an error correction cycle on a set of $n$ $\ket{0}$'s for the
stabilizer generated by $M_1, \ldots M_{n-k}, \Zbar_1, \ldots, \Zbar_k$.
This stabilizer has $n$ generators, so there is only one joint $+1$
eigenvector, which is just the encoded $0$ for the original code.