\appendix

\chapter{Quantum Gates}
\label{app-gates}

It is usually helpful to think of a quantum computer as performing a series
of {\em gates}, drawn from some fairly small basic set of physically
implementable unitary transformations.  The net transformation applied
to the quantum computer is the product of the unitary transformations
associated with the gates performed.  In order to have a universal quantum
computer, it should be possible to get arbitrarily close to any unitary
transformation.  This property makes no guarantees about how many gates
are required to get within $\epsilon$ of the desired unitary operation,
and figuring out how to get a given operator with the minimum
number of basic gates is the goal of quantum algorithm design.

There are a number of known sets of universal quantum gates
\cite{lloyd-universal, gates}.  For instance, all single-qubit unitary
operators and the controlled-NOT together comprise a universal set.  The
controlled-NOT gate (or CNOT) is a two-qubit operator that flips the second
qubit iff the first qubit is $\ket{1}$.  It has the matrix
\begin{equation}
	\pmqty{1 & 0 & 0 & 0 \\ 0 & 1 & 0 & 0 \\ 0 & 0 & 0 & 1 \\ 0 & 0 & 1 & 0}.
\end{equation}
In fact, the controlled-NOT and one single-qubit operator are sufficient,
as long as the the single-qubit rotation acts by an angle incommensurate
with $2 \pi$.  Another finite universal set of quantum gates consists of the
Hadamard rotation $R$,
\begin{equation}
	R = \frac{1}{\sqrt{2}} \pmqty{1 & \ 1 \\ 1 & -1},
\end{equation}
the phase gate $P$,
\begin{equation}
	P = \pmqty{1 & 0 \\ 0 & i},
\end{equation}
the controlled-NOT, and the Toffoli gate, which is a three-qubit gate which
flips the third qubit iff the first two qubits are in the state $\ket{11}$.

In addition to the gates mentioned above, I refer to a number of other simple
gates in this thesis.  For instance, the simple NOT gate, the sign gate, and
the combined bit and sign flip gate (which are equal to $\X$, $\Z$, and $\Y$,
respectively) play a crucial role in the stabilizer formalism.  I also
refer to two other single-qubit gates related to $P$ and $R$.  They are
\begin{equation}
	Q = \frac{1}{\sqrt{2}} \pmqty{\ 1 & \ i \\ -i & -1},
\end{equation}
and
\begin{equation}
	T = \frac{1}{\sqrt{2}} \pmqty{ 1 & -i \\ 1 & \ i }.
\end{equation}
I also occasionally refer to the ``conditional sign'' gate, which is a
two-qubit gate that gives the basis state $\ket{11}$ a sign of $-1$ and
leaves the other three basis states alone.  The conditional sign gate is
equivalent to the controlled-NOT via conjugation of one qubit by $R$.  The
conditional sign gate is effectively a controlled-$\Z$ gate, where $\Z$
gets applied to one qubit iff the other qubit is $\ket{1}$.  I also use
an analogous controlled-$\Y$ operator.  The CNOT is the controlled-$\X$.

To describe a series of gates, it is usually helpful to draw a diagram
of the gate array.  Horizontal lines represent the qubits of the quantum
computer, which enter at the left and leave from the right.  A summary
of the symbols I use for the various gates is given in figure
\ref{fig-gates}.
\begin{figure}
	\centering
	\begin{picture}(320, 160)

		\put(35,150){\line(1,0){20}}
		\put(45,150){\circle{8}}
		\put(45,146){\line(0,1){8}}
		\put(55,144){\makebox(50,12){$\X$}}

		\put(130,150){\line(1,0){4}}
		\put(146,150){\line(1,0){4}}
		\put(134,144){\framebox(12,12){$\Y$}}
		\put(150,144){\makebox(50,12){$\Y$}}

		\put(230,150){\line(1,0){4}}
		\put(246,150){\line(1,0){4}}
		\put(234,144){\framebox(12,12){$\Z$}}
		\put(250,144){\makebox(50,12){$\Z$}}

		\put(35,110){\line(1,0){20}}
		\put(35,90){\line(1,0){20}}
		\put(45,110){\circle*{4}}
		\put(45,110){\line(0,-1){24}}
		\put(45,90){\circle{8}}
		\put(0,104){\makebox(35,12){Control}}
		\put(0,84){\makebox(35,12){Target}}
		\put(55,94){\makebox(70,12){Controlled-NOT}}

		\put(130,110){\line(1,0){20}}
		\put(130,90){\line(1,0){4}}
		\put(146,90){\line(1,0){4}}
		\put(140,110){\circle*{4}}
		\put(140,110){\line(0,-1){14}}
		\put(134,84){\framebox(12,12){$\Y$}}
		\put(150,94){\makebox(70,12){Controlled-$\Y$}}

		\put(230,110){\line(1,0){20}}
		\put(230,90){\line(1,0){4}}
		\put(246,90){\line(1,0){4}}
		\put(240,110){\circle*{4}}
		\put(240,110){\line(0,-1){14}}
		\put(234,84){\framebox(12,12){$\Z$}}
		\put(250,94){\makebox(70,12){Controlled-$\Z$}}

		\put(35,50){\line(1,0){20}}
		\put(35,30){\line(1,0){20}}
		\put(35,10){\line(1,0){20}}
		\put(45,50){\circle*{4}}
		\put(45,30){\circle*{4}}
		\put(45,50){\line(0,-1){44}}
		\put(45,10){\circle{8}}
		\put(0,44){\makebox(30,12){Control}}
		\put(0,24){\makebox(30,12){Control}}
		\put(0,4){\makebox(30,12){Target}}
		\put(55,24){\makebox(70,12){Toffoli gate}}

		\put(130,50){\line(1,0){4}}
		\put(146,50){\line(1,0){4}}
		\put(134,44){\framebox(12,12){$P$}}
		\put(150,44){\makebox(50,12){$P$}}

		\put(230,50){\line(1,0){4}}
		\put(246,50){\line(1,0){4}}
		\put(234,44){\framebox(12,12){$Q$}}
		\put(250,44){\makebox(50,12){$Q$}}

		\put(130,10){\line(1,0){4}}
		\put(146,10){\line(1,0){4}}
		\put(134,4){\framebox(12,12){$R$}}
		\put(150,4){\makebox(70,12){Hadamard $R$}}

		\put(230,10){\line(1,0){4}}
		\put(246,10){\line(1,0){4}}
		\put(234,4){\framebox(12,12){$T$}}
		\put(250,4){\makebox(50,12){$T$}}

	\end{picture}
	\caption{Various quantum gates.}
	\label{fig-gates}
\end{figure}

\chapter{Glossary}
\label{app-glossary}

\begin{description}

	\item[additive code] Another name for a stabilizer code.  Often contrasted
	with linear quantum codes, which are a subclass of additive codes.

	\item[amplitude damping channel] A channel for which the $\ket{1}$ state
	may relax to the $\ket{0}$ state with some probability.  An example is
	a two-level atom relaxing via spontaneous emission.

	\item[cat state] The $n$-qubit entangled state $\ket{0 \ldots 0} + \ket{1
	\ldots 1}$.  Cat states act as ancillas in many fault-tolerant operations.

	\item[coding space] The subset of the Hilbert space corresponding to correctly
	encoded data.  The coding space forms a Hilbert space in its own right.

	\item[concatenation] The process of encoding the physical qubits making
	up one code as the logical qubits of a second code.  Concatenated codes
	are particularly simple to correct, and can be used to perform arbitrarily
	long fault-tolerant computations as long as the physical error rate is
	below some threshhold.

	\item[CSS code] Short for Calderbank-Shor-Steane code.  A CSS code is formed
	from two classical error-correcting codes.  CSS codes can easily take
	advantage of results from the theory of classical error-correcting codes
	and are also well-suited for fault-tolerant computation.  See sections
	\ref{sec-stab-examples} and \ref{sec-CSS}.

	\item[cyclic code] A code that is invariant under cyclic permutations of
	the qubits.

	\item[decoherence] The process whereby a quantum system interacts with its
	environment, which acts to effectively measure the system.  The world looks
	classical at large scales because of decoherence.  Decoherence is likely to
	be a major cause of errors in quantum computers.

	\item[degenerate code] A code for which linearly independent correctable
	errors acting on the coding space sometimes produce linearly dependent states.
	Degenerate codes bypass many of the known bounds on efficiency of quantum
	codes and have the potential to be much more efficient than any
	nondegenerate code.

	\item[depolarizing channel] A channel that produces a random error on each
	qubit with some fixed probability.

	\item[distance] The minimum weight of any operator $E_a^\dagger E_b$ such that
	equation~(\ref{eq-condition}) is {\em not} satisfied for an orthonormal
	basis of the coding space.  A quantum code with distance $d$ can detect
	up to $d-1$ errors, or it can correct $\lfloor (d-1)/2 \rfloor$ general
	errors or $d-1$ located errors.

	\item[entanglement] Nonlocal, nonclassical correlations between two quantum
	systems.  The presence of entangled states gives quantum computers their
	additional computational power relative to classical computers.

	\item[entanglement purification protocol] Often abbreviated EPP.  An EPP is
	a protocol for producing high-quality EPR pairs from a larger number of
	low-quality EPR pairs.  EPPs are classified depending on whether they use
	one-way or two-way classical communication.  A 1-way EPP (or 1-EPP) is
	equivalent to a quantum error-correcting code.

	\item[EPR pair] Short for Einstein-Podalsky-Rosen pair.  An EPR pair is
	the entangled state $(1/ \sqrt{2}) \left(\ket{00} + \ket{11} \right)$, and
	acts as a basic unit of entanglement.

	\item[erasure channel] A channel that produces one or more located errors.

	\item[error syndrome] A number classifying the error that has occurred.  For
	a stabilizer code, the error syndrome is a binary number with a 1 for each
	generator of the stabilizer the error anticommutes with and a 0 for each
	generator of the stabilizer the error commutes with.

	\item[fault-tolerance] The property (possessed by a network of gates) that an
	error on a single physical qubit or gate can only produce one error in any
	given block of an error-correcting code.  A fault-tolerant network can be
	used to perform computations that are more resistant to errors than the
	physical qubits and gates composing the computer, provided the error rate
	is low enough to begin with.  A valid fault-tolerant operation should also
	map the coding space into itself to avoid producing errors when none existed
	before.

	\item[leakage error] An error in which a qubit leaves the allowed computational
	space.  By measuring each qubit to see if it is in the computational space,
	a leakage error can be converted into a located error.

	\item[linear code] A stabilizer code that, when described in the GF(4)
	formalism (section~\ref{sec-alternate}), has a stabilizer that is invariant
	under multiplication by $\omega$.  Often contrasted with an additive code.

	\item[located error] Sometimes called an erasure.  A located error is an
	error which acts on a known qubit in an unknown way.  A located error is
	easier to correct than a general error acting on an unknown qubit.

	\item[nice error basis] A basis which shares certain essential properties
	with the Pauli matrices and can be used to define a generalized stabilizer
	code.  See section~\ref{sec-qudits}.

	\item[nondegenerate code] A code for which linearly independent correctable
	errors acting on the coding space always produce linearly independent states.
	Nondegenerate codes are much easier to set bounds on than degenerate codes.

	\item[pasting] A construction for combining two quantum codes to make a
	single larger code.  See section~\ref{sec-construction}.

	\item[perfect code] A code for which every error syndrome corresponds to a
	correctable error.  See section~\ref{sec-perfect} for a construction of
	the distance three perfect codes.

	\item[quantum error-correcting code] Sometimes abbreviated QECC.  A QECC
	is a set of states that can be restored to their original state after some
	number of errors occur.  A QECC must satisfy equation~(\ref{eq-condition}).

	\item[qubit] A single two-state quantum system that serves as the fundamental
	unit of a quantum computer.  The word ``qubit'' comes from ``quantum bit.''

	\item[qudit] A $d$-dimensional generalization of a qubit.

	\item[shadow] The set of operators in $\G$ which commute with the
	even-weight elements of the stabilizer and anticommute with the odd-weight
	elements of the stabilizer.

	\item[shadow enumerator] The weight enumerator of the shadow.  It is useful
	for setting bounds on the existence of quantum codes.

	\item[stabilizer] The set of tensor products of Pauli matrices that fix
	every state in the coding space.  The stabilizer is an Abelian subgroup
	of the group $\G$ defined in section~\ref{sec-general-prop}.  The stabilizer
	contains all of the vital information about a code.  In particular, operators
	in $\G$ that anticommute with some element of the stabilizer can be
	detected by the code.

	\item[stabilizer code] A quantum code that can be described by giving its
	stabilizer.  Also called an additive code or a GF(4) code.

	\item[teleportation] A process whereby a quantum state is destroyed and exactly
	reconstructed elsewhere.  Quantum teleportation of a single qubit requires
	one EPR pair shared between the source and destination, and involves two
	measurements on the source qubit.  The two bits from the measurements must be
	classically transmitted to the destination in order to reconstruct the original
	quantum state.

	\item[threshhold] The error rate below which a suitably configured quantum
	computer can be used to perform arbitrarily long computations.  Current
	methods for proving the existence of a threshhold use concatenated codes.
	Most estimates of the threshhold lie in the range $10^{-6}$ -- $10^{-4}$.

	\item[transversal operation] An operation applied in parallel to the
	various qubits in a block of a quantum error-correcting code.  Qubits
	from one block can only interact with corresponding qubits from another
	block or from an ancilla.  Any transversal operation is automatically
	fault-tolerant.

	\item[weight] A property of operators only defined on operators which
	can be written as the tensor product of single-qubit operators.  For such
	an operator, the weight is the number of single-qubit operators in the
	product that are not equal to the identity.

	\item[weight enumerator] A polynomial whose coefficients $c_n$ are the number
	of elements of weight $n$ in some set, such as the stabilizer or
	the normalizer of the stabilizer.  Weight enumerators are very helpful
	in setting bounds on the possible existence of quantum error-correcting
	codes through identities such as the quantum MacWilliams identities (equation
	(\ref{eq-QMW})).

\end{description}