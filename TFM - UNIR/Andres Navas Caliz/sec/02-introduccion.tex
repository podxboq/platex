\chapter{Introducción}

\section{Motivación}

En un mundo cada vez mas interconectado donde las personas, empresas, estados y organismos internacionales son conscientes de la necesidad de optimizar recursos para a su vez maximizar resultados, se observa que la capacidad para resolver problemas de optimización de forma eficiente se vuelve crucial. En este contexto, dominar técnicas que permitan identificar las soluciones óptimas en tiempo real o cercano a tiempo real es fundamental para mantenerse competitivo y adaptarse rápidamente a entornos cambiantes. La posibilidad de optimizar procesos, minimizar costos o maximizar rendimientos no solo impulsa la eficiencia, sino que también puede tener un impacto significativo en la rentabilidad y la sostenibilidad. Por lo tanto, la habilidad para resolver problemas de optimización de manera eficaz se convierte en una ventaja estratégica en un mundo donde la eficiencia y la agilidad son clave para el éxito. Dentro de la inmensidad de problemas de optimización, existe un subgrupo de estos, denominados problemas de tipo grafo. \\

Dentro de esta clase de problemas podemos encontrar algunos tan famosos e importantes como el problema Max Cut, el problema Maximum Independent Set o el problema Traveling Salesman Problem, TSP por sus siglas en ingles. Además de ser problemas de optimización centrales en muchas de las actividades humanas como el transporte o las telecomunicaciones, estos poseen una característica clave. El problema Max Cut o el TSP son problemas NP-Duros, lo cual significa que no se posee un algoritmo determinista que sea eficiente en la tarea resolver el problema. Existe incluso la posibilidad de que pueda no existir un algoritmo no estocástico capaz de resolver de forma eficiente esta clase de problemas.\\

Es en este contexto la computación cuántica emerge como potencial solución a la necesidad de optimizar recursos, ya sea en forma de tiempo, costo o calidad de la solución encontrada. Dentro del campo de la computación cuántica, una de las ramas mas extensas, desarrolladas e impulsadas es aquella en la cual se utilizan las particularidades de este paradigma de computación para el desarrollo de algoritmos capaces de resolver problemas de optimización que superen en rendimiento a los algoritmos análogos clásicos. 

\newpage

Dado que los problemas que se pretenden resolver son NP-Duros, los algoritmos clásicos que se poseen para resolver dichos problemas son no deterministas y por lo tanto no se garantiza que la solución que se encuentra sea una solución óptima. Es decir, no se garantiza que los algoritmos que se usan para resolver esta clase de problemas generen soluciones suficientemente buenas. Para poder resolver clásicamente y de forma eficiente estos problemas, se recurren a heurísticas, lo que convierten a los algoritmos en no deterministas. En el contexto de la algoritmia, heurística hace referencia al uso de reglas generales o prácticas no derivadas analíticamente del problema específico las cuales permiten encontrar soluciones aproximadamente óptimas o sub-óptimas. Poder encontrar nuevas heurísticas basadas en las propiedades únicas de la mecánica cuántica o desarrollar posibles algoritmos cuánticos deterministas que permitan encontrar soluciones que sean mejores que aquellas que se obtienen mediante los algoritmos clásicos, es el objetivo principal dentro de la rama de computación cuántica aplicada a resolver problemas de optimización. 

\section{Planteamiento del problema}

En la actualidad, existen numerosos algoritmos cuánticos diseñados para abordar problemas de optimización, conocidos como algoritmos cuánticos variacionales o VQA (por sus siglas en inglés). Sin embargo, estos algoritmos adolecen de diversos inconvenientes, tanto económicos como técnicos. En términos económicos, estos algoritmos requieren un uso intensivo de computadoras cuánticas, lo que hace que su ejecución en entornos de producción sea poco rentable debido al alto costo, actualmente, asociado con el uso de estas plataformas. Desde el punto de vista técnico, estos algoritmos sufren de problemas como las barren plateaus, especialmente cuando el tamaño del problema y, por ende, el número de hiperparámetros a optimizar aumenta. Esto conlleva un mayor riesgo de quedar atrapado en mínimos locales que representan soluciones subóptimas del problema. Otro desafío técnico proviene de las plataformas cuánticas en las que se ejecutan estos algoritmos. En la actualidad, estas maquinas experimentan niveles significativos de ruido que limitan considerablemente la eficiencia y eficacia de los algoritmos cuánticos variacionales. Con el objetivo de abordar las deficiencias presentes en este tipo de algoritmos cuánticos, se han desarrollado técnicas de preprocesamiento que mejoran el rendimiento en términos de calidad, velocidad de convergencia y reducción del coste económico para su ejecución. 

\newpage

La idea fundamental de estas estrategias implica utilizar los algoritmos cuánticos solo en la etapa final del proceso de optimización para obtener la solución óptima al problema. Una de las tecnologías mas adecuadas para este propósito es el uso de tensor networks como herramienta de preprocesamiento. Bajo este formalismo, se pueden aprovechar algoritmos de inspiración cuántica que abordan los problemas actuales asociados con los algoritmos cuánticos variacionales, mitigando así sus limitaciones. \\


En particular, hay propuestas especificas de esquemas de trabajo donde se combina el algoritmo DMRG, el cual es un algoritmo de optimización de tensor networks y un protocolo que permite traducir las soluciones obtenidas por dicho algoritmo a circuitos cuánticos parametrizados, usados posteriormente en los VQA`s. Esto permite inicializar los algoritmos variacionales en estados cuánticos que representan soluciones de partida mas favorables. En una primer paso, se realiza una optimización clásica usando el algoritmo DMRG. Este algoritmo realiza una primera optimización clásica del problema, generando como resultados, estados cuánticos que representan soluciones suboptimas pero mejores que los estados de partida aleatorios. Posteriormente, el estado generado por el algoritmo DMRG, se utiliza como punto de partida para la segunda parte de la optimización usando algoritmos de tipo VQA. Esta combinación de algoritmos clásicos y cuánticos producen mejores resultados que los obtenidos por cada uno de ellos de forma separada. \\

El esquema general propuesto sin embargo adolece de problemas que hacen inviable su uso como método de preprocesado cuando consideramos la resolución de problemas clásicos como el problema TSP o Max Cut, entre otros. El comportamiento del algoritmo cuántico variacional, es errático y no consigue mejorar el resultado obtenido por el algoritmo DMRG, el cual es el que se ha utilizado para inicializar el algoritmo VQE. La razón principal de este hecho, reside en que la optimización de problemas clásicos usando el algoritmo DMRG conduce a estados cuánticos que no poseen entropía de coherencia. La consecuencia de esto, es que los algoritmos cuánticos variacionales, empiezan en mínimos locales de estados producto en los cuales se quedan atrapados y por tanto, no se puede mejorar la solución de partida utilizando algoritmos variacionales. 

\newpage

\section{Estructura del trabajo}

En este proyecto se pretende constatar dos hechos, principalmente. En primer lugar, que las metodologías que combinan algoritmos de tensor networks y VQA's presentes en el estado del arte fallan cuando se trata de aplicar dichas metodologías a la resolución de problemas clásicos. Dado que los esquemas generales propuestos fallan en problemas clásicos, se propone un esquema mejorado que supere los problemas presentes en los trabajos anteriores. Este nuevo esquema implica usar nuevos algoritmos de tensor networks para la parte de optimización clásica, como lo son el algoritmo MPO time evolution o el algoritmo imaginary time evolution. El uso de este nuevo enfoque de algoritmos de optimización de tensor networks evita, tanto en problemas clásicos como cuánticos, el caer en estados producto que representen mínimos locales. Esto mitiga el hecho de que el algoritmo variacional cuántico se quede atrapado. \\

Por otro lado, se demuestra la hipótesis de que los estados cuánticos de partida que maximizan el rendimiento de los algoritmos cuánticos variacionales, son los denominados estados de Gibbs. La hipótesis de que estos estados son aquellos que maximizan el rendimiento, se debe a que son este tipo de estados los que minimizan la energía del problema al mismo tiempo que poseen un grado de superposición máxima. Demostrar este punto implica demostrar a su vez, que los mejores algoritmos de optimización de tensor networks, o de cualquier otro tipo, que se pueden usar como algoritmos de preprocesado son aquellos que realizan la optimización en base a creación de los estados de Gibbs. \\

Bajo esta propuesta, se mejora en el avance de la computación cuántica para conseguir una ventaja cuántica real en problemas de interés industrial. También se presentan resultados numéricos, los cuales buscan dar respuesta a por que los estados de Gibbs maximizan el rendimiento de los algoritmos variacionales cuánticos. \\

El documento, en los capítulos siguientes, se divide en, un capitulo del estado de la técnica donde se presentan los paradigmas de computación de tensor networks y de computación cuántica. En este capitulo se presenta toda la información previa, relativa al trabajo desarrollado, que se considera relevante para entender las novedades aportadas en este proyecto. 

\newpage

Un capitulo de objetivos, en el cual se enumeran los objetivos y las hipótesis planteadas así como la metodología seguida. En el capitulo de desarrollo del trabajo, presentamos la implementación y procedimiento llevado a cabo para poder obtener los resultados que validen las hipótesis planteadas. El capitulo posterior, está dedicado a la discusión de los resultados obtenidos. En el penúltimo, en el capitulo de conclusiones se exponen las ideas finales, de forma sintetizada, relacionando los resultados presentados con las hipótesis planteadas inicialmente. El ultimo capitulo, llamado trabajo futuro, se exponen las lineas de continuación que dan continuidad al proyecto iniciado en este trabajo.

