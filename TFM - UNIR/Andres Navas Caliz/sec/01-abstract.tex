\chapter{Abstract}

Optimization of problems, particularly the optimization of NP-Hard problems, is a field of great interest. Some of the problems in this field are central to the activities of many companies, organizations, and institutions. \\

This work focuses on solving NP-Hard problems using a combination of two computing paradigms: quantum computing and tensor networks. The approach of combining two different computing paradigms aims to enhance and mitigate the deficiencies currently present in quantum computing. On one hand, tensor networks are a form of computing based on efficient operations and algorithms that run on classical platforms, which possess great technological maturity. On the other hand, quantum computing is a computing paradigm that employs novel operations, algorithms, and protocols executed on quantum platforms, which currently are in a state of low technological maturity. \\

This document studies existing pre-optimization techniques, in which tensor networks and variational quantum algorithms are combined, and then demonstrates that these techniques fail when applied to solving classical problems. Subsequently, a new preprocessing scheme is proposed that surpasses its predecessors, based on the construction of the so-called pure Gibbs quantum states, which are states that, at a given energy, possess maximum coherence entropy. Numerical analyses are conducted and presented to validate the work, comparing the proposed novel scheme with its predecessor. \\



{\bf Keywords:} tensor networks, quantum computing, Gibbs states, DMRG, MPO Time Evolution, VQA.