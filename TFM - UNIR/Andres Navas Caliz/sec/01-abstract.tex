\chapter{Abstract}

The optimisation of problems, in particular the optimisation of NP-hard problems, is a field of study of great interest. Some of the problems in this field are central to the activity of many companies, organisations and institutions. \\

This work focuses on solving NP-hard problems using a combination of two computing paradigms, quantum computing and tensor networks. The approach of combining two different computing paradigms aims to improve and mitigate the current shortcomings of quantum computing. On the one hand, tensor networks is a form of computing based on efficient operations and algorithms running on classical platforms, which are technologically mature. On the other hand, quantum computing is a computing paradigm that employs novel operations, algorithms and protocols running on quantum platforms, which are currently in a state of low technological maturity. \\

In this document we review existing preprocessing techniques, which combine tensor networks and variational quantum algorithms, and then prove that these techniques fail when applied to classical problem solving. Subsequently, a new preprocessing scheme is proposed that outperforms its predecessors, based on the construction of so-called quantum Gibbs states, which maximise the performance of quantum variational algorithms. Numerical analyses are performed and presented to validate the work, in which the proposed nobel scheme is compared with its predecessor. \\



{\bf Keywords:} tensor networks, quantum computing, Gibbs states, DMRG, MPO time evolution, VQA's.