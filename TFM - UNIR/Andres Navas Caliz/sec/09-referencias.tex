\begin{thebibliography}{a}

%1
\bibitem[Yang, 2020]{yang} 
Yang, Q., Li, Y., Huang, P. (2020). A novel formulation of the max-cut problem and related algorithm. Journal of Combinatorial Optimization.
\url{https://www.sciencedirect.com/science/article/abs/pii/S0096300319309622}

%2
\bibitem[Heal, 2022]{heal} 
Heal, M., Dashtipour, K., Gogate, M. (2022). Formulations and algorithms to find maximal and maximum independent sets of graphs. Journal of Graph Theory.
\url{https://american-cse.org/csci2022-ieee/pdfs/CSCI2022-2lPzsUSRQukMlxf8K2x89I/202800a517/202800a517.pdf}

%3
\bibitem[Bermejo, 2021]{bermejo} 
González Bermejo, S., Alonso-Linaje, G., Atchade-Adelomou, P. (2021). GPS: A new TSP formulation for its generalizations type QUBO. Journal of Optimization Theory and Applications, 189(1), 203-227.
\url{https://arxiv.org/abs/2110.12158}

%4
\bibitem[Hoerl, 2021]{hoerl}
Hoerl, R., Jensen, W. (2021). Understanding and addressing complexity in problem solving. Quality Engineering, 33(1), 1-15.
\url{https://www.researchgate.net/publication/354042853_Understanding_and_addressing_complexity_in_problem_solving}

%5
\bibitem[Shannon, 1948]{Shannon1948} 
Shannon, C. E. (1948). A Mathematical Theory of Communication. The Bell System Technical Journal, 27, 379–423, 623–656. (Reprinted with corrections from July, October, 1948).

%6
\bibitem[Benioff, 1980]{benioff}
Benioff, P. (1980). The computer as a physical system: A microscopic quantum mechanical Hamiltonian model of computers as represented by Turing machines. Journal of Statistical Physics, 22(5), 563-591. \url{https://doi.org/10.1007/bf01011339}

%7
\bibitem[Feynman, 1982]{richard}
Feynman, R. P. (1982). Simulating physics with computers. International Journal of Theoretical Physics, 21(6-7), 467-488. \url{https://doi.org/10.1007/bf02650179}

%8 
\bibitem[Deutsch, 1992]{deutsch}
Deutsch, D., Jozsa, R. (1992). Rapid solution of problems by quantum computation. Proceedings of the Royal Society of London. Series A: Mathematical and Physical Sciences, 439(1907), 553-558.

%9
\bibitem[Shor, 2002]{shor}
Shor, P. W. (2002). Algorithms for quantum computation: discrete logarithms and factoring. Proceedings 35th Annual Symposium on Foundations of Computer Science.

%10
\bibitem[Dirac, 1939]{dirac1939}
Dirac, P. A. M. (1939). A new notation for quantum mechanics. Proceedings of the Cambridge Philosophical Society, 35(4), 416-418. \url{https://www.ifsc.usp.br/~lattice/wp-content/uploads/2014/02/Dirac_notation.pdf}

%11
\bibitem[Heusler, 2020]{bloch}
Heusler, S., Schlummer, P., \& Ubben, M. S. (2020). A knot theoretic extension of the Bloch sphere representation for qubits in Hilbert space and its application to contextuality and many-worlds theories. Symmetry, 12(7), 1135. \url{https://doi.org/10.3390/sym12071135}

%12
\bibitem[Montañez, 2023]{montañez}
Montañez-Barrera, J. A., Willsch, D., Maldonado-Romo, A., Michielsen, K. (2023). Unbalanced penalization: A new approach to encode inequality constraints of combinatorial problems for quantum optimization algorithms.

%13
\bibitem[Peruzzo, 2013]{peruzzo}
Peruzzo, A., McClean, J., Shadbolt, P., Yung, M.-H., Zhou, X.-Q., Love, P. J., Aspuru-Guzik, A., O’Brien, J. L. (2013). A variational eigenvalue solver on a quantum processor. \url{https://arxiv.org/abs/1304.3061}

%14
\bibitem[Farhi, 2014]{farhi}
Farhi, E., Goldstone, J. (2014). A Quantum Approximate Optimization Algorithm. \url{https://arxiv.org/abs/1411.4028}

%15
\bibitem[Jack , 2020]{jack}
PennyLane. (s.f.). Quantum Machine Learning Demos: Tutorial on QAOA (Quantum Approximate Optimization Algorithm) Introduction. Recuperado de \url{https://pennylane.ai/qml/demos/tutorial_qaoa_intro/}

%16
\bibitem[Amaro, 2022]{amaro}
Amaro, D., Modica, C., Rosenkranz, M., Fiorentini, M., Benedetti, M., Lubasch, M. (2022). Filtering variational quantum algorithms for combinatorial optimization. Physical Review A, 105(1), 012405. \url{https://arxiv.org/abs/2106.10055}

%17
\bibitem[Luis, 2023]{luis}
Díez-Valle, P., Luis-Hita, J., Hernández-Santana, S., Martínez-García, F., Díaz-Fernández, Á., Andrés, E., García-Ripoll, J. J., Sánchez-Martínez, E., Porras, D. (2023)
\url{https://arxiv.org/abs/2302.04196}

%18
\bibitem[Zhu, 2022]{zhu}
Zhu, L., Tang, H. L., Barron, G. S., Calderon-Vargas, F. A., Mayhall, N. J., Barnes, E., Economou, S. E. (2022). An Adaptive Quantum Approximate Optimization Algorithm for Solving Combinatorial Problems on a Quantum Computer. Physical Review A, 106(1), 012403. \url{https://arxiv.org/abs/2005.10258}

%19
\bibitem[McClean, 2018]{mcClean}
McClean, J. R., Boixo, S., Smelyanskiy, V. N., Babbush, R., Neven, H. (2018). Barren plateaus in quantum neural network training landscapes. Google Inc..
\url{https://arxiv.org/abs/1803.11173}

%20
\bibitem[Orús, 2014]{orus}
Orús, R. (2014). A Practical Introduction to Tensor Networks: Matrix Product States and Projected Entangled Pair States. Institute of Physics, Johannes Gutenberg University, Mainz, Germany.
\url{https://arxiv.org/abs/1306.2164}

%17
\bibitem[Tensor Network, n.d.]{tn}
Tensor Network. (n.d.). Matrix product operator (MPO). Recuperado de
\url{https://tensornetwork.org/mpo/}

%21
\bibitem[Fröwis, 2018]{fröwis}
Fröwis, F., Nebendahl, V., Dür, W. (2018). Tensor operators: Constructions and applications for long-range interaction systems. Institut für Theoretische Physik, Universität Innsbruck.
\url{https://arxiv.org/abs/1003.1047}

%22
\bibitem[Rudolph, 2023]{manuel}
Rudolph, M. S., Miller, J., Motlagh, D., Chen, J., Acharya, A., Perdomo-Ortiz, A. (2023). Synergy between quantum circuits and tensor networks: Short-cutting the race to practical quantum advantage. Zapata Computing.
\url{https://arxiv.org/abs/2208.13673}

%23
\bibitem[Dborin, 2021]{dborin}
Dborin, J., Barratt, F., Wimalaweera, V., Wright, L., Green, A. G. (2021). Matrix product state pre-training for quantum machine learning. London Centre for Nanotechnology, University College London.
\url{https://arxiv.org/abs/2106.05742}

%24
\bibitem[Huggins, 2018]{huggins}
Huggins, W., Patil, P., Mitchell, B., Whaley, K. B., Stoudenmire, E. M. (2018). Towards quantum machine learning with tensor networks. University of California Berkeley.
\url{https://arxiv.org/abs/1803.11537}

%25
\bibitem[Schollwöck, 2011]{schollwöck}
Schollwöck, U. (2011). The density-matrix renormalization group in the age of matrix product states. Department of Physics, Arnold Sommerfeld Center for Theoretical Physics and Center for NanoScience, University of Munich.
\url{https://arxiv.org/abs/1008.3477}

%26
\bibitem[Ran, 2020]{ran}
Ran, S.-J. (2020). Encoding of matrix product states into quantum circuits of one- and two-qubit gates. Department of Physics, Capital Normal University.
\url{https://arxiv.org/abs/1908.07958}

%27
\bibitem[Shirakawa, 2021]{shirakawa}
Shirakawa, T., Ueda, H.,  Yunoki, S. (2021). Automatic quantum circuit encoding of a given arbitrary quantum state. Computational Materials Science Research Team, RIKEN Center for Computational Science (R-CCS).
\url{https://arxiv.org/abs/2112.14524}

%28
\bibitem[Rudolph, 2022]{rudolph}
Rudolph, M. S., Chen, J., Miller, J., Acharya, A.,  Perdomo-Ortiz, A. (2022). Decomposition of matrix product states into shallow quantum circuits. Zapata Computing Canada Inc.
\url{https://arxiv.org/abs/2209.00595}

%29
\bibitem[Tucci, 2008]{tucci}
Tucci, R. R. (2008). An introduction to Cartan’s KAK decomposition for QC programmers. P.O. Box 226, Bedford, MA.
\url{https://arxiv.org/abs/quant-ph/0507171}

%30
\bibitem[Cenedese, 2023]{cenedese}
Cenedese, G., Bondani, M., Rosa, D., Benenti, G. (2023). Generation of pseudo-random quantum states on actual quantum processors.
\url{https://www.mdpi.com/1099-4300/25/4/607}

%31
\bibitem[Mori, 2018]{mori}
Mori, T., Ikeda, T. N., Kaminishi, E., Ueda, M. (2018). Thermalization and prethermalization in isolated quantum systems: A theoretical overview. Department of Physics, Graduate School of Science, The University of Tokyo.
\url{https://arxiv.org/abs/1712.08790}

%32
\bibitem[Xi, 2015 ]{xi}
Xi, Z., Li, Y., Fan, H. (2015 ). Quantum coherence and correlations in quantum systems.
\url{https://arxiv.org/abs/1408.3194}

%33
\bibitem[Polkovnikov, 2010 ]{polkovnikov}
Polkovnikov, A. (2010). Microscopic diagonal entropy and its connection to basic thermodynamic relations. Department of Physics, Boston University.
\url{https://arxiv.org/abs/0806.2862}

%34
\bibitem[Riera, 2018 ]{riera}
Bera, M. N., Riera, A., Lewenstein, M., Khanian, Z. B., Winter, A. (2018). Thermodynamics as a consequence of information conservation. ICFO – Institut de Ciencies Fotoniques, The Barcelona Institute of Science and Technology.
\url{https://arxiv.org/abs/1707.01750}

%35
\bibitem[Zaletel, 2014 ]{zaletel}
Zaletel, M. P., Mong, R. S. K., Karrasch, C., Moore, J. E., Pollmann, F. (2014). Time-evolving a matrix product state with long-ranged interactions. Physical Review B, 90(4), Article 045113.
\url{https://arxiv.org/abs/1407.1832}

\end{thebibliography}
%\bibliographystyle{plain}
%\bibliography{bibliografia}