\begin{thebibliography}{a}

%1

\bibitem[Shannon(1948)]{Shannon1948} 
Shannon, C. E. (1948). A Mathematical Theory of Communication. The Bell System Technical Journal, 27, 379–423, 623–656. (Reprinted with corrections from July, October, 1948).

%2

\bibitem[Benioff, 1980]{benioff}
Benioff, P. (1980). The computer as a physical system: A microscopic quantum mechanical Hamiltonian model of computers as represented by Turing machines. Journal of Statistical Physics, 22(5), 563-591. \url{https://doi.org/10.1007/bf01011339}

%3

\bibitem[Feynman, 1982]{richard}
Feynman, R. P. (1982). Simulating physics with computers. International Journal of Theoretical Physics, 21(6-7), 467-488. \url{https://doi.org/10.1007/bf02650179}

%4 

\bibitem[Deutsch, 1992]{deutsch}
Deutsch, D., Jozsa, R. (1992). Rapid solution of problems by quantum computation. Proceedings of the Royal Society of London. Series A: Mathematical and Physical Sciences, 439(1907), 553-558.


%5

\bibitem[Shor, 2002]{shor}
Shor, P. W. (2002). Algorithms for quantum computation: discrete logarithms and factoring. Proceedings 35th Annual Symposium on Foundations of Computer Science.

%6

\bibitem[Dirac, 1939]{dirac1939}
Dirac, P. A. M. (1939). A new notation for quantum mechanics. Proceedings of the Cambridge Philosophical Society, 35(4), 416-418. \url{https://www.ifsc.usp.br/~lattice/wp-content/uploads/2014/02/Dirac_notation.pdf}

%7

\bibitem[Heusler, 2020]{bloch}
Heusler, S., Schlummer, P., \& Ubben, M. S. (2020). A knot theoretic extension of the Bloch sphere representation for qubits in Hilbert space and its application to contextuality and many-worlds theories. Symmetry, 12(7), 1135. \url{https://doi.org/10.3390/sym12071135}

%8

\bibitem[Montañez, 2023]{montañez}
Montañez-Barrera, J. A., Willsch, D., Maldonado-Romo, A., Michielsen, K. (2023). Unbalanced penalization: A new approach to encode inequality constraints of combinatorial problems for quantum optimization algorithms.

%9

\bibitem[Peruzzo, 2013]{peruzzo}
Peruzzo, A., McClean, J., Shadbolt, P., Yung, M.-H., Zhou, X.-Q., Love, P. J., Aspuru-Guzik, A., O’Brien, J. L. (2013). A variational eigenvalue solver on a quantum processor. \url{https://arxiv.org/abs/1304.3061}

%10

\bibitem[Farhi, 2014]{farhi}
Farhi, E., Goldstone, J. (2014). A Quantum Approximate Optimization Algorithm. \url{https://arxiv.org/abs/1411.4028}

%11

\bibitem[Jack , 2020]{jack}
PennyLane. (s.f.). Quantum Machine Learning Demos: Tutorial on QAOA (Quantum Approximate Optimization Algorithm) Introduction. Recuperado de \url{https://pennylane.ai/qml/demos/tutorial_qaoa_intro/}

%12

\bibitem[Amaro, 2022]{amaro}
Amaro, D., Modica, C., Rosenkranz, M., Fiorentini, M., Benedetti, M., Lubasch, M. (2022). Filtering variational quantum algorithms for combinatorial optimization. Physical Review A, 105(1), 012405. \url{https://arxiv.org/abs/2106.10055}

%13

\bibitem[Luis, 2023]{luis}
Díez-Valle, P., Luis-Hita, J., Hernández-Santana, S., Martínez-García, F., Díaz-Fernández, Á., Andrés, E., García-Ripoll, J. J., Sánchez-Martínez, E., Porras, D. (2023)
\url{https://arxiv.org/abs/2302.04196}

%14

\bibitem[Zhu, 2022]{zhu}
Zhu, L., Tang, H. L., Barron, G. S., Calderon-Vargas, F. A., Mayhall, N. J., Barnes, E., Economou, S. E. (2022). An Adaptive Quantum Approximate Optimization Algorithm for Solving Combinatorial Problems on a Quantum Computer. Physical Review A, 106(1), 012403. \url{https://arxiv.org/abs/2005.10258}

%15

\bibitem[Orús, 2014]{orus}
Orús, R. (2014). A Practical Introduction to Tensor Networks: Matrix Product States and Projected Entangled Pair States. Institute of Physics, Johannes Gutenberg University, Mainz, Germany.
\url{https://arxiv.org/abs/1306.2164}

\end{thebibliography}
%\bibliographystyle{plain}
%\bibliography{bibliografia}