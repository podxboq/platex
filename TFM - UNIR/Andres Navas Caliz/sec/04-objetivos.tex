\chapter{Objetivos}

\section{Objetivo general}

El objetivo principal de este proyecto es el desarrollo de un esquema de preprocesado basado en tensor networks que permita superar las deficiencias presentes en los protocolos existentes. El nuevo esquema de preprocesado está basado en la creación de estados de Gibbs puros.

\section{Objetivos específicos}

Los objetivos planteados para el trabajo han sido los siguientes:

\begin{itemize}
    
    \item Comprender e implementar algoritmos de tensor networks que puedan ser de interés en el campo de la computación cuántica.
    
    \item Implementar un método híbrido de tensor networks y computación cuántica para la resolución de problemas de optimización.
   
    \item Implementar un método híbrido novedoso, basado en los estados cuánticos de Gibbs, que supere a los presentes en el estado del arte.
     
    \item Realizar comparativas de rendimiento entre el método propuesto y el método presente en el estado del arte.

\end{itemize}

\section{Hipótesis planteadas}

La hipótesis planteada para este trabajo y que recoge los objetivos anteriormente mencionados es la siguiente:\\

\begin{mdframed}[backgroundcolor=black!10]
\centering 

Los estados cuánticos de inicialización que maximizan el rendimiento de los algoritmos cuánticos variacionales o VQA´s son los estados cuánticos de Gibbs. 

\end{mdframed}