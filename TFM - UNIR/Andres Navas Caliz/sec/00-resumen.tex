\chapter{Resumen}

La optimización de problemas, en particular la optimización de problemas NP-Duros, es un campo de estudio de gran interés. Algunos de los problemas presentes en este campo, son centrales dentro de la actividad de muchas empresas, organismos e instituciones. \\

Este trabajo se centra en la resolución de problemas NP-Duros utilizando la combinación de dos paradigmas de computación, la computación cuántica y tensor networks. El planteamiento de combinar dos paradigmas de computación diferentes tiene como objetivo mejorar y mitigar las deficiencias que, en la actualidad, posee la computación cuántica. Por un lado, tensor networks es una forma de computación que se basa en operaciones y algoritmos eficientes que se ejecutan en plataformas clásicas, las cuales poseen una gran madurez tecnológica. Por otro lado, la computación cuántica es un paradigma de computación que emplea operaciones, algoritmos y protocolos novedosos, que se ejecutan en plataformas cuánticas, las cuales en la actualidad, se encuentran en un estado de poca madurez tecnológica. \\

En este documento se realiza un estudio de las técnicas de pre-optimización existentes, en las cuales se combina tensor networks y algoritmos cuánticos variacionales, para posteriormente comprobar que dichas técnicas fallan cuando se tratan de aplicar en la resolución de problemas clásicos. Posteriormente, se propone un nuevo esquema de preprocesado que supera a sus antecesores, basado en la construcción de los denominados estados cuánticos de Gibbs puros, los cuales son estados que a una energía dada, poseen entropía de coherencia máxima. Se realizan y presentan análisis numéricos para validar el trabajo realizado, en el cual se compara el esquema novedoso propuesto con su antecesor. \\

{\bf Palabras clave:} tensor networks, computación  cuántica, estados de Gibbs, DMRG, MPO Time Evolution, VQA.