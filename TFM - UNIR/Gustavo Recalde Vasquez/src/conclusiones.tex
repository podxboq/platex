\chapter{Conclusiones y trabajo futuro}

Se deberá proporcionar una evaluación de la ventaja de la solución y su aplicabilidad para resolver el problema propuesto.

\section{Conclusiones}

Este último capítulo (en ocasiones, dos capítulos complementarios) es habitual en todos los tipos de trabajos y presenta el resumen final de tu trabajo y debe servir para informar del alcance y relevancia de tu aportación.
Suele estructurarse empezando con un resumen del problema tratado, de cómo se ha abordado y de por qué la solución sería válida.

Es recomendable que incluya también un resumen de las contribuciones del trabajo, en el que relaciones las contribuciones y los resultados obtenidos con los objetivos que habías planteado para el trabajo, discutiendo hasta qué punto has conseguido resolver los objetivos planteados.

\section{Líneas de trabajo futuro}

Finalmente, se suele dedicar una última sección a hablar de líneas de trabajo futuro que podrían aportar valor añadido al TFE realizado. La sección debería señalar las perspectivas de futuro que abre el trabajo desarrollado para el campo de estudio definido. En el fondo, debes justificar de qué modo puede emplearse la aportación que has desarrollado y en qué campos.