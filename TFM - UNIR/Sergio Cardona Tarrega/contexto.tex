\chapter{Contexto y estado del técnica}


\section{Antecedentes históricos}
La conjetura de Collatz, también conocida por otros nombres como conjetura de Siracusa, conjetura de Ulam o conjetura $3n+1$, es uno de los problemas matemáticos sin resolver más conocido hasta la fecha.

Fue enunciada por Lothar Collatz en 1937 y el 7 de julio de 2021.
$\\$

La empresa japonesa Bakuage Co. ofreció un premio de 120 millones de yenes (aproximadamente 1.085.000 dólares estadounidenses) a quien aporte una prueba matemática o un contraejemplo que la confirme o la desmienta.
$\\$

Muchos matemáticos han tratado de resolver esta conjetura sin éxito y muchos de ellos desaconsejan incluso el intentarla siquiera.

Por ejemplo Paul Erdös, manifestó en una ocasión que las matemáticas no estaban listas para este tipo de problemas.
\begin{quote}
    "Mathematics may not be ready for such problems."
\end{quote}


\subsection{Vida de Lothar Collatz}
Lothar Collatz fue un matemático alemán nacido el 6 de julio de 1910 en Arnsberg y fallecido en Varna, Bulgaria el 26 de septiembre de 1990 a la edad de 80 años.
$\\$

Collatz estudió en diferentes universidades alemanas incluyendo la Universidad de Berlín bajo la tutela de Alfred Klose y realizó el doctorado en 1935 con una tésis titutalada "Das Differenzenverfahren mit höherer Approximation für Differentialgleichungen" (El método de diferencias finitas para una mayor aproximación para ecuaciones diferenciales lineales).
Trabajó en distintas universidades como la Universidad Técnica de Darmstadt o la Universidad de Hamburgo. Luego de su jubilación como profesor, participó activamente en congresos hasta que finalmente falleció inesperadamente de un ataque al corazón en Varna (Bulgaria) el 26 de septiembre de 1990, mientras asistía a un congreso.
$\\$

Por su gran trabajo, a Lothar Collatz le fueron entregados numerosos galardones, entre los cuales están:
\begin{enumerate}
    \item Elegido miembro de la Academia alemana de las ciencias naturales Leopoldina, la Academia de Ciencias del Instituto de Boloña y la Academia de pmoderna en Italia.
    \item Miembro honorario de la Sociedad Matemática de Hamburgo
    \item Grados honoríficos otorgados por la Universidad de São Paulo, la Universidad Técnica de Viena, la Universidad de Dundee (Escocia), la Universidad Brunel (Inglaterra), la Universidad de Hannover y la Universidad Politécnica de Dresde.
\end{enumerate}






\section{Definición de la conjetura}

\subsection{Función de Collatz}

Definimos la Función de Collatz como una función $C:\mathbb{N}\mapsto\mathbb{N}$ con:

\begin{equation}
\label{C(n)}
    C(n) = \begin{cases}
    3n+1 & \text{si n impar},\\
    \frac{n}{2} & \text{si n par}.
    \end{cases}
\end{equation}

\subsection{Sucesión de Collatz}

Si partimos de un elemento $n \in \mathbf{N}$ podemos definir Sucesión de Collatz como una sucesión:

\begin{equation}
\label{SucesionCollatz}
    a_i = \begin{cases}
    n & \text{para } i=0,\\
    C(a_{i-1}) & \text{para } i>0.
    \end{cases}
\end{equation}

es decir $a_0 = n$, $a_1 = C(n)$ y así sucesivamente.


Escribiremos de forma más compacta $a_i = C^i(n)$ como aplicar la Función de Collatz $C(C(...(C(n)))$ $i$ veces (Notar que $C^0(n) = n$ sería aplicar la Función de Collatz $0$ veces).


A menudo también se le denomina órbita o trayectoria de $n$ al conjunto formado por:
$$ \{ n, C(n), C^2(n), ... \} $$

Por ejemplo si tomamos $n=7$, tendremos que su Sucesión de Collatz es:
$$\{7, 22, 11, 34, 17, 26, 13, ... \}$$

\subsection{Conjetura de Collatz}
La conjetura de Collatz afirma que para cualquier natural $n\in\mathbf{N}$, si aplicamos sucesivamente la Función de Collatz, eventualmente alcanzaremos el número $1$.

Dicho de otra forma: $\forall n \in \mathbf{N}$ $\exists k\in \mathbf{N}$ tal que $a_k = 1$.

Notar que una vez alcanzado $a_k=1$, se cumplirá que $a_{k+1}=4$, $a_{k+2}=2$ y $a_{k+3}=1$ nuevamente, por tanto estos 3 valores se irán repitiendo eternamente a partir de entonces.
$\\$

La demostración de la conjetura puede dividirse en 2 partes:
\begin{enumerate}
    \item Demostrar que no existe otro $n \in \mathbf{N}$ además de $1$ que verifica que $n=T^k(n)$ para algún $k \in \mathbf{N}$ (a lo que llamaremos formar un ciclo).
    \item Demostrar que no existe ningún número natural que diverja al infinito, es decir $\nexists n \in \mathbf{N}$ tal que $\lim\limits_{i \to \infty}T^i(n)=\infty$.
\end{enumerate}

De encontrar un $n$ que no verifique cualquiera de estas dos propiedades, habríamos encontrado un contraejemplo a dicha conjetura.
$\\$

El primer enfoque que le daremos a este trabajo consistirá en desarrollar un algoritmo cuántico que busque contraejemplos de determinados números que forman un ciclo y que sea más eficiente que cualquier posible alternativa clásica.

En el segundo enfoque trataremos de utilizar la superposición cuántica para desarrollar un algoritmo cuántico que compute iteraciones de Collatz simultáneamente en todos los números de determinados dígitos en su representación binaria.

\section{Estado actual}

En esta sección enumeraremos resultados y definiciones sobre la conjetura recogidos en (\cite{lagariasAnnotatedBibliography1, lagariasAnnotatedBibliography2}) a no ser que especifiquemos lo contrario.



\subsection{Definición de Función de Collatz con atajo}

Definimos la Función de Collatz con atajo $T:\mathbf{N} \mapsto \mathbf{N}$ con:

\begin{equation}
\label{T(n)}
    T(n) = \begin{cases}
    \frac{3n+1}{2} & \text{si n impar},\\
    \frac{n}{2} & \text{si n par}.
    \end{cases}
\end{equation}

Que sería equivalente a:

$$
    T(n) = \begin{cases}
    C(C(n)) & \text{si n impar},\\
    C(n) & \text{si n par}.
    \end{cases}
$$

Es decir estaríamos aplicando la función $C(n)$ dos veces en el caso que $n$ fuera impar, y sólo una vez en el caso de que $n$ fuera par.

Esta equivalencia se debe a que si $n$ es impar entonces $3n+1$ es par y por tanto $C(3n+1)=\frac{3n+1}{2}$.


Bajo esta nueva definición de Función de Collatz, la conjetura de Collatz aseguraría pues, que para todo $n \in \mathbf{N}$ existe algún $k \in \mathbf{N}$ tal que $T^k(n)=1$

De ahora en adelante, cuando hablemos de Función de Collatz, nos referiremos a esta nueva definición de Función de Collatz con atajo.


\subsection{Definición Sucesión de Collatz con atajo}
Análogamente a como definíamos la Sucesión de Collatz para la función de Collatz $C(n)$, definimos ahora Sucesión de Collatz con atajo como una sucesión $b_i$ de forma que:

\begin{equation}
\label{SucesionCollatzAtajada}
    b_i = \begin{cases}
    n & \text{para } i=0,\\
    T(b_{i-1}) & \text{para } i>0.
    \end{cases}
\end{equation}

Bajo esta nueva sucesión, la conjetura aseguraría también que $\forall n \in \mathbf{N}$ $\exists k \in \mathbf{N}$ tal que $b_k = 1$.

Una vez alcanzado $b_k=1$, se cumplirá ahora que $b_{k+1} = 2$ y $b_{k+2}=1$, y estos dos valores se repetirán eternamente formando lo que denominaremos un Ciclo de Collatz de longitud 2

De ahora en adelante, cuando hablemos de Sucesión de Collatz, nos referiremos a esta nueva definición de Sucesión de Collatz con atajo.


\subsection{Definición Ciclo de Collatz}
Dado un número natural $n\in  \mathbf{N}$, diremos que forma un Ciclo de Collatz de longitud $l\in\mathbf{N}$, si la sucesión finita formada por:
$$ \{ n, T(n), T^2(n), ..., T^{l-1}(n)\} $$
Verifica que $T^i(n) \neq T^j(n)$ si $i \equiv j \pmod l $

Consideraremos además que $n< T^i(n)$, $\forall i \in [1,l[$ (Notar que siempre podemos reordenar los elementos del conjunto para obtener el elemento más pequeño en primer lugar).

De ser cierta la conjetura de Collatz, implicaría (entre otras cosas) que el único Ciclo de Collatz existente es el de longitud $2$ formado por $\{ 1, 2 \}$.

Es evidente que al verificarse $T^l(n)=n$ entonces también se verificará $T^{l+1}(n) = T(n)$, ya que $T^{l+1}(n)=T(T^l(n))=T(n)$.

También se verificará $T^{l+2}(n)=T^2(n), T^{l+3}(n)=T^3(n)...$ y en general:
$$i \equiv j \pmod l \Longrightarrow T^i(n) = T^j(n)$$


\subsection{Definición Sucesión de Paridad}
Dada una Sucesión de Collatz  $\{b_0, b_1, ... \} = \{ n, T(n), T^2(n), ... \}$, definimos Sucesión de Paridad de la Sucesión de Collatz como la sucesión:
$$
    \{ p_0, p_1, p_2, ...\}
$$
Donde $\forall i \in \mathbf{N}$:
$$
    p_i = \begin{cases}
    1 & \text{si $T^i(n)$ impar},\\
    0 & \text{si $T^i(n)$ par}.
    \end{cases}
$$

Es decir definimos Sucesión de Paridad de una Sucesión de Collatz como la sucesión en la que cada elemento es la paridad de dicha Sucesión de Collatz.

Por ejemplo si tomamos $n=7$, y su respectiva Sucesión de Collatz $\{7, 11, 17, 26, 13, ... \}$, entonces su Sucesión de Paridad vendrá dada por:
$$\{ 1, 1, 1, 0, 1, 0, 0, 1,... \}$$


\subsection{Definición Vector de Paridad}
Análogamente, dado un Ciclo de Collatz de longitud $l$: $\{ n, T(n), T^2(n), ..., T^{l-1}(n)\}$, definiremos Vector de Paridad de dicho Ciclo de Collatz como el vector:

$$
    v=(v_0, v_1, v_2, ..., v_{l-1}) \in \{0,1\}^l
$$

Donde $\forall i \in [0,l[$:
$$
    v_i = \begin{cases}
    1 & \text{si $T^i(n)$ impar},\\
    0 & \text{si $T^i(n)$ impar}.
    \end{cases}
$$

Y de nuevo es evidente que se verificará:
$i \equiv j \pmod l \Longrightarrow v_i = v_j$. 


\subsection{Extensión racional}
Podemos extender el dominio de la función $T(n)$ (\ref{T(n)}) a los racionales con denominador impar, ya que $T$ está bien definida en $\mathbf{Q}[(2)]$ como se muestra en (\cite{Lagarias1990}).

De esta manera obtenemos nuestra Función de Collatz racional:

\begin{equation}
\label{T(q)}
    T(q) = \begin{cases}
    \frac{3q+1}{2} & \text{si el numerador de q es impar},\\
    \frac{q}{2} & \text{si el numerador de q es par}.
    \end{cases}
\end{equation}

Y tendremos también que su respectiva Sucesión de Collatz racional sería:

\begin{equation}
\label{SucesionCollatzRacional}
    c_i = \begin{cases}
    q & \text{para } i=0,\\
    T(c_{i-1}) & \text{para } i>0.
    \end{cases}
\end{equation}

Y su vector de paridad racional asociado sería un vector $v=(v_0, v_1, ..., v_{l-1}) \in \{0, 1\}^l$  tal que $\forall i \in [0,l[$:

$$
    v_i = \begin{cases}
    1 & \text{si $T^i(n)$ impar},\\
    0 & \text{si $T^i(n)$ impar}.
    \end{cases}
$$
$\\$


Definimos ahora $x_l(v)$ como una función $x_l : \{0, 1\}^l \mapsto \mathbf{Q}[(2)]$ tal que:
\begin{equation}
    \label{FormulaCicloCollatz}
    x_l(v)=\frac{\sum\limits_{j=0}^{l-1} v_j 2^j 3^{v_{j+1}+ ... + v_{l-1}}}{2^l-3^{v_0+...+v_{l-1}}}
\end{equation}

Notar que esta expresión surge al tomar un elemento arbitrario $q \in \mathbf{Q}[(2)]$ y desarrollar las sucesivas aplicaciones de $T(q),T(T(q)), ...T^{l-1}(q)$ teniendo en cuenta que la paridad de los numeradores de dichos elementos viene dada por el vector $(v_0, v_1, ..., v_{l-1})$
$\\$

Entonces dado un vector $v=(v_0, v_1, ..., v_{l-1})$, tomando $q=x(v)$, tendríamos que $q$ forma un ciclo de Collatz racional de longitud $l$: $\{ q, T(q), ..., T^{l-1}(q)\}$, con vector de paridad racional asociado $v$. Que además será único, es decir dos vectores de paridad diferentes generarán dos racionales con denominador impar diferentes.
$\\$

Por tanto, encontrar un Ciclo de Collatz de longitud $l>2$, para así encontrar un contraejemplo de la conjetura, se reducirá a encontrar un Ciclo de Collatz Racional $\{ q, T(q), ..., T^{l-1}(q)\}$ (y que por tanto verifique \ref{FormulaCicloCollatz}), que además verifique que $q\in \mathbf{N}$.

Dicho de otra manera, se reducirá a encontrar un elemento $v=(v_0, v_1, ..., v_{l-1}) \in \{0, 1\}^l$ de forma que $2^l > 3^{v_0 + ... + v_{l-1}}$ y que además $2^l - 3^{v_0 + ... + v_{l-1}}$ divida a $\sum\limits_{j=0}^{l-1} v_j 2^j 3^{v_{j+1}+ ... + v_{l-1}}$.
$\\$

Encontrar un Ciclo de Collatz de longitud arbitraria $l\in\mathbf{N}$, se puede ver también según (\cite{Lagarias1990}) como encontrar $x, y \in \mathbf{N}$ con $l=x+y$, y ciertos $a_0, a_1, ..., a_{y-1} \in \mathbf{N}$ con $l>a_0>a_1>...>a_{y-1}\geq0$ de forma que se verifique:
$$\frac{\sum\limits_{j=0}^{y-1}3^j2^{a_j}}{2^{x+y} - 3^y} \in \mathbf{N}$$
Y por tanto que el denominador sea positivo y divida al numerador.




\subsection{Definición k-Ciclo de Collatz}
Sea un Ciclo de Collatz de longitud $l$: $\{ n, T(n), T^2(n), ..., T^{l-1}(n)\}$ con paridad de Ciclo $\{v_0, v_1, ..., v_{l-1}\}$, se dice que es un k-Ciclo de Collatz si existen exactamente $k$ elementos $v_i\in\{v_0, v_1, ..., v_{l-1}\}$ tal que $v_i \equiv 2 \pmod4$.

Es decir, existen exactamente $k$ elementos $T^i(n)$ que son pares pero no divisibles por $4$.

Notar que el hecho de que $T^i(n)$ sea múltiplo de 2 y no de 4, equivale a que su correspondiente elemento en el vector de paridad $v_i$ sea 1, pero en cambio el siguiente elemento del vector de paridad $v_{i+1}$ sea 0 (o que $v_{l-1}=1$ y $v_0=0$ en caso de tratarse del último elemento).
$\\$

Podemos entonces agrupar los elementos del ciclo en $k$ bloques donde cada bloque i-ésimo contendrá $x_i$ elementos del Ciclo de Collatz impares y $y_i$ elementos pares. Entonces $x_0$ significará que el vector de paridad $v$ tiene sus primeros $x_0$ elementos con valor 1, los siguientes $y_0$ valores serán 0, los siguientes $x_1$ valores serán 1 y así sucesivamente.

Reescribiríamos entonces el k-Ciclo de Collatz como:
$$\{ (x_1,y_1),(x_2,y_2),...,(x_{k},y_{k}) \}$$
con $x_i \in \mathbf{N}$, $y_i \in \mathbf{N} \bigcup \{0\}$
$\\$

Podemos también extender la definición a los racionales de denominador impar partiendo de un Ciclo de Collatz Racional. Diremos entonces que se trata de un k-Ciclo de Collatz Racional.
$\\$


Por ejemplo el Ciclo de Collatz Racional de longitud $6$ y Vector de Paridad $(1,1,0,1,0,0)$ correspondería según la fórmula \ref{FormulaCicloCollatz} al racional:
$$q = \frac{1\cdot2^0\cdot3^2 + 1\cdot2^1\cdot3^1 + 0\cdot2^2\cdot3^1 + 1\cdot2^3\cdot3^0 + 0\cdot2^4\cdot3^0 + 0\cdot2^5\cdot3^0}{2^6-3^3}=\frac{9+6+8}{64-27}=\frac{23}{37}$$



Que forma el ciclo:
$$\{ \frac{23}{37}, \frac{53}{37}, \frac{98}{37}, \frac{49}{37}, \frac{92}{37}, \frac{46}{37}\}$$

Este Ciclo de Collatz Racional podría reescribirse como un 2-Ciclo de Collatz con $x_1=2$, $y_1=1$, $x_2=1$, $y_2=2$:
$$\{(2,1),(1,2)\}$$

Bajo esta nueva definición, se verificará que $l=\sum\limits_{j=0}^{k-1} x_j + \sum\limits_{j=0}^{k-1}y_j$ será la longitud del Ciclo de Collatz.
$\\$

Podemos por comodidad, definir $x:=\sum\limits_{j=0}^{k-1} x_j$ como la cantidad de términos impares en el Ciclo de Collatz y también $y:=\sum\limits_{j=0}^{k-1} y_j$ como la cantidad de términos pares, obteniendo así la relación $l=x+y$.



\subsection{Fórmula 1-Ciclo de Collatz}
En párticular, para un 1-Ciclo de Collatz $\{ (x_1, y_1) \}$, tendremos $x=x_1$, $y=y_1$, y por tanto $l=x+y=x_1+y_1$.
La fórmula \ref{FormulaCicloCollatz} nos quedaría como:
\begin{equation}
    \label{Formula1CicloCollatz}
    q = \frac{3^y-2^x}{2^{x+y}-3^y}
\end{equation}
Que puede reescribirse también como:
\begin{equation}
    \label{Formula1CicloCollatzAlterna}
    q = -1 + 2^x\frac{2^y-1}{2^{x+y}-3^y}
\end{equation}

Por lo tanto la búsqueda de 1-Ciclos con valores en $\mathbf{Z}$, se reduciría a encontrar $x,y \in \mathbf{N}$ tal que $2^{x+y}-3^y$ divida a $2^y-1$ (puesto que no divide a $2^x$).

Si además imponemos la condición de que $2^{x+y}>3^y$ estaremos buscando 1-Ciclos de Collatz en $\mathbf{N}$.
$\\$


Está sin embargo demostrado (\cite{LowerBoundsCycleLength}) que el único 1-Ciclo en $\mathbf{N}$ (y de hecho el único k-ciclo con $k\leq91$) es el 1-Ciclo $\{(1,0)\}$ correspondiente a nuestro ya conocido ciclo $\{1,2\}$ de longitud 2.
$\\$

También está demostrado (\cite{Lagarias1990}) que los únicos Ciclos de Collatz en $\mathbf{Z}$ con longitud menor de $250.000$ son:
\begin{enumerate}
    \item $\{ {1, 2} \}$ con paridad $(1, 0)$
    \item $\{ {0} \}$ con paridad $(0)$
    \item $\{ {-1} \}$ con paridad $(1)$
    \item $\{ {-5, -7, -10} \}$ con paridad $(1, 1, 0)$
    \item $\{ {-17, -25, -37, -55, -82, -41, -61, -91, -136, -68, -34} \}$ con paridad $(1, 1, 1, 1, 0, 1, 1, 1, 0, 0, 0)$
\end{enumerate}

Los ciclos generados por $0$ y por $-1$ podemos considerarlos 0-ciclos o puntos estables ya que son ciclos de un solo elemento.

Los ciclos generados por $1$ y por $-5$ serían 1-ciclos de la forma $\{ (1,1) \}$ y $\{ (2,1) \}$.

El ciclo generado por $-17$ sería un 2-ciclo de la forma $\{ (4,1), (3,3) \}$

Como podemos observar el 2-ciclo generado por $-17$ es el mismo conjunto que el 2-ciclo generado por $-41$ y lo representaríamos con $-17$ como primer elemento al ser el menor de ambos (en valor absoluto).



\subsection{Fórmula de un 2-Ciclo de Collatz}
En párticular, para un 2-Ciclo de Collatz $\{ (x_1, y_1),(x_2,y_2) \}$, la fórmula \ref{FormulaCicloCollatz} nos quedaría como:

\begin{equation}
    \label{Formula2CicloCollatz}
        q = \frac{3^{x_1+x_2} - 2^{x_1}3^{x_2} + 2^{x_1+y_1}3^{x_2} - 2^{x_1+y_1+x_2}  }{2^{x_1+y_1+x_2+y_2}-3^{y_1+y_2} }
\end{equation}

Ahora tomando por comodidad $l_1 = x_1 + y_1$, $l_2 = x_2 + y_2$, $l=l_1+l_2$, $x=x_1+x_2$, $y=y_1+y_2$ y sacando $-1$ de la fracción y agrupando términos obtendríamos:

\begin{equation}
    \label{Formula2CicloCollatzAlterna}
        q = -1 + 3^{x_2}2^{x_1} \frac{2^{y_1} - 1}{2^l-3^x} + 2^{l_1}2^{x_2} \frac{2^{y_2}-1}{2^l-3^x}
\end{equation}
$\\$

Si consiguiéramos de alguna manera encontrar $x_1,y_1,x_2,y_2 \in \mathbf{N}$ de forma que $2^l-3^x$ divida simultáneamente a $2^{y_1}-1$ y a $2^{y_2}-1$ tendríamos automáticamente un 2-ciclo.

Si bien podrían existir otros 2-ciclos que no verificaran lo anterior, una estrategia de búsqueda podría ser realizar un algoritmo que iterara sobre estas variables en busca de esta divisibilidad.




\subsection{Fórmula de un k-Ciclo de Collatz}
Si en general partimos de un k-Ciclo de Collatz con $k$ arbitrario: $\{ (x_1, y_1), ..., (x_{k},y_{k}) \}$, la fórmula \ref{FormulaCicloCollatz} nos quedaría como:

\begin{equation}
    \label{FormulakCicloCollatz}
        q = \frac{\sum\limits_{j=1}^k 3^{X-X_j}2^{L_{j-1}}(3^{x_j}-2^{x_j})}{2^L-3^X} 
\end{equation}

Donde hemos tomado por comodidad:

$X_j = \sum\limits_{i=1}^j x_i$ ($X_0=0, X_1=x_1$, $X_2=x_1+x_2$,...)

$L_j = \sum\limits_{i=1}^j x_i + y_i$ ($L_0=0, L_1=x_1+y_1$, $L_2=x_1+y_1+x_2+y_2$,...)

Y también para ahorrarnos el subíndice:

$X=X_k$

$L=L_k$
$\\$

Análogamente a los casos particulares anteriores, podemos extraer $-1$ de la fracción y arreglar términos para obtener una nueva ecuación:


\begin{equation}
    \label{FormulakCicloCollatzAlterna}
        q = -1 + \sum_{j=1}^k 3^{X-X_j} 2^{L_{j-1}} 2^{x_j} \frac{2^{y_j}-1}{2^L-3^X}
\end{equation}
$\\$

Si tomamos $k=1$ o $k=2$ obtendríamos nuestras fórmulas previas: \ref{Formula1CicloCollatz}, \ref{Formula1CicloCollatzAlterna}, \ref{Formula2CicloCollatz}, \ref{Formula2CicloCollatzAlterna}

Si por ejemplo tomáramos $k=3$, obtendríamos:

$X_1=x_1$
$X_2=x_1+x_2$
$X=X_3=x_1+x_2+x_3$
$L_0=0$
$L_1=x_1+y_1$
$L_2=x_1+y_1+x_2+y_2$
$L=L_3=x_1+y_1+x_2+y_2+x_3+y_3$

y por tanto la fórmula \ref{FormulakCicloCollatzAlterna} nos quedaría:
$$q = -1 + 3^{X-X_1}2^{L_0}2^{x_1}\frac{2^{y_1}-1}{2^L-3^X} + 3^{X-X_2}2^{L_1}2^{x_2}\frac{2^{y_2}-1}{2^L-3^X} + 3^{X-X}2^{L_2}2^{x_3}\frac{2^{y_3}-1}{2^L-3^X}$$

Que remplazando nos quedaría finalmente:
$$q = -1 + 3^{x_3+x_2}2^{x_1}\frac{2^{y_1}-1}{2^L-3^X} + 3^{x_3}2^{x_1+y_1+x_2}\frac{2^{y_2}-1}{2^L-3^X} + 2^{x_1+y_1+x_2+y_2+x_3}\frac{2^{y_3}-1}{2^L-3^X}$$
$\\$

Así pues, una estrategia para buscar k-Ciclos de Collatz en general (con valores en $\mathbf{N}$), podría ser hacer un algoritmo que itere sobre las variables $L$ y $X$ y buscar valores $y_1$, $y_2$,... de forma que se verifique la divisibilidad de cada uno de los $2^{y_i}-1$ entre el denominador común $2^L-3^X$.

Más adelante en el documento, detallaremos como podemos utilizar el algoritmo de Shor para abordar el problema con computación cuántica siguiendo esta estrategia.




\subsection{Tiempo de parada}
Dada una Sucesión de Collatz $\{a_0, a_1, ...\}$ con $a_i$ definidos según \ref{SucesionCollatz}, definimos tiempo de parada $\sigma(n)$ como el menor $k$ de forma que $T^k(n)<n$, y se define como $\infty$ si dicho $k$ no existe.

Una estrategia para abordar matemáticamente la conjetura de Collatz sería demostrar que dicho $k$ existe siempre para cualquier $n$, dado que así la conjetura podría ser probada por inducción.




\subsection{Tiempo de parada total}
Dada una Sucesión de Collatz $\{a_0, a_1, ...\}$ con $a_i$ definidos según \ref{SucesionCollatz}, definimos tiempo de parada total $\sigma_{\infty}(n)$ como el menor $k$ de forma que $T^k(n)=1$.

Dicho de otra manera $\sigma_{\infty}(n):= inf\{ k: T^k(n) = 1\}$.
$\\$

Probar que esté $k$ existe siempre equivaldría literalmente a demostrar la conjetura de Collatz.



\subsection{Tiempo de parada total escalado}
Dada una Sucesión de Collatz $\{a_0, a_1, ...\}$ con $a_i$ definidos según \ref{SucesionCollatz}, definimos tiempo de parada total escalado como $\gamma(n):=\frac{\sigma_{\infty}(n)}{log n}$



\subsection{Cotas inferiores en el tiempo de parada}
Existen infinitos enteros positivos cuyo tiempo de parada total esta acotado inferiormente:
$$\sigma_{\infty}(n)>6.14316 \log n$$
(\cite{LowerBoundsStoppingTime}).

En este mismo artículo, se afirma creer que la cota podría mejorarse a $6.95212 \log n$ pero se necesitaría más capacidad de computación.



\subsection{Restricción en la longitud de un Ciclo}
Dado un Ciclo de Collatz de longitud $l$, dicha longitud deberá verificar:

\begin{equation}
    \label{RestriccionLongitudCiclo}
    l = 301 994a + 17 087 915b + 85 137 581c
\end{equation}
con $a,b,c \in \mathbf{Z}$, $a,c\geq 0$, $b>0$ y $ac=0$ (\cite{LowerBoundsCycleLength}).
$\\$

Podemos usar esta restricción para agilizar enormemente nuestros algoritmos de búsqueda de Ciclos de Collatz, tanto clásicos como cuánticos.




\subsection{Inexistencia de k-Ciclos con $k\leq91$}
Hasta la fecha, el mejor resultado que tenemos sobre la inexistencia de k-Ciclos de Collatz (exceptuando el ciclo trivial $\{1,2\}$ es que no existen k-Ciclos de Collatz con $k\leq91$ (\cite{hercher2023collatzmcycles}).
$\\$

Podemos utilizar esto para optimizar todavía más nuestros algoritmos de búsqueda de Ciclos de Collatz, descartándolos directamente si $k\leq91$.



\subsection{Comprobación computacional}
En 2021, se comprobó computacionalmente que todos los valores hasta $2^{68} \approx 2.95 \times 10^{20}$ convergen a $1$, es decir que la conjetura es cierta para todos los $n\in\mathbf{N}$ tal que $n\leq2^{68}$ (\cite{Barina2021}).



\subsection{Cota para casi todas las órbitas}
En 2019, Terrence Tao publicó lo que hasta hoy se conoce como el resultado más cercano a probar la conjetura sin llegar a demostrarla (\cite{Tao_2022}).
$\\$

Tao demostró utilizando densidad logarítmica, que dada una función $f:\mathbf{N}\longrightarrow\mathbf{R}$ con $\lim_{n \to \infty} f(n) = +\infty$, casi todas las órbitas de Collatz cumplen que $T_{min}(n) \leq f(n)$, donde se define $T_{min}(n):=\inf_{k\in\mathbf{N}}T^k(n)$.



\subsection{Representación binaria}
Una forma de afrontar el problema, es mediante representación binaria.

Escribimos un numero natural $n\in\mathbf{N}$ en su representación binaria $n_{q-1}n_{q-2}...n_2n_1n_0$, con $n_i \in \{0,1\} \forall i \in [0,q[$.

Esto es $n=n_{q-1}2^{q-1} + n_{q-2}2^{q-2} ... + 2^2n_2 + 2n_1 + n_0$.
$\\$

Entonces un número será par si $n_0=0$ y será impar si $n_0=1$.
$\\$

Dividir un número entre 2, no es más que eliminar el dígito $n_0$, desplazando así todos los dígitos a la derecha.
Dicho de otra manera, $\frac{n}{2} = n_{q-1}2^{q-2} + n_{q-2}2^{q-3} + ... + 2^2n_3 + 2n_2 + n_1$.
De igual manera, multiplicar por 2 un número $n$, consistirá en añadirle un digito $0$ a la derecha. Si en lugar de un $0$, añadieramos un $1$ estaríamos realizando la operación $2n+1$
$\\$

Siguiendo esta idea, podemos descomponer la operación $3n+1$ como una suma entre un número $n$ representado en binario, y el mismo número con $1$ al final, ya que $3n+1=n + (2n+1)$.
$\\$

Por tanto un posible algoritmo binario para realizar una iteración de Collatz podría ser el siguiente:

\begin{enumerate}
    \item Si el último dígito es 0, lo eliminamos desplazando todos los demás dígitos a la derecha (dividimos entre 2).
    \item Si el último dígito es 1, entonces realizamos una suma binaria entre dicho número y dicho número con un 1 añadido a la derecha (multiplicamos por 3 y sumamos 1).
\end{enumerate}

Este algoritmo representaría una operación de Collatz $C(n)$ definida en $\ref{C(n)}$.

Podríamos repetir el proceso hasta llegar a $n=1$ para verificar que dicho $n$ alcanza el $1$.
$\\$

Por ejemplo si $n=13$, su representación binaria seria $n=1101$ y el algoritmo quedaría de la siguiente manera:
\begin{center}
    \begin{tabular}{|cccccc|c|}
            \hline
          &   & 1 & 1 & 0 & 1       & 13    \\
        + & 1 & 1 & 0 & 1 & 1       &       \\
            \hline
        1 & 0 & 1 & 0 & 0 & 0       & 40    \\
            \hline
          & 1 & 0 & 1 & 0 & 0       & 20    \\
            \hline
          &   & 1 & 0 & 1 & 0       & 10    \\
            \hline
          &   &   & 1 & 0 & 1       & 5     \\
          & + & 1 & 0 & 1 & 1       &       \\
            \hline
          & 1 & 0 & 0 & 0 & 0       & 16    \\
            \hline
          &   & 1 & 0 & 0 & 0       & 8     \\
            \hline
          &   &   & 1 & 0 & 0       & 4     \\
                      \hline
          &   &   &   & 1 & 0       & 2     \\
                      \hline
          &   &   &   &   & 1       & 1     \\
        \hline
    \end{tabular}
\end{center}
$\\$

Nuestro segundo enfoque consistirá en realizar un algoritmo cuántico que realice este proceso simultáneamente con todos los números de $q$ dígitos binarios, representados por $q$ qubits.