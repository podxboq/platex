\chapter{Resultados / Análisis / Comparativa / Discusión de resultados}
Es muy importante presentar los resultados de forma clara y detallada. Analizar estos resultados, presentar comparaciones con otros trabajos y/o entre las distintas soluciones que aportemos. También es muy importante analizar cuidadosamente todos estos resultados.




BUSCAR errores segun aumenta el numero de qubit
IBM: da el porcentaje de error de sus ordenadores segun numero de qubits


Shor para factorizar numero de 10 digitos añadiendo los qubit de correccion de error.
Comparo las 3 propuestas.
eso minimizara la correccion de errores

cuanta tarda en ejecutarse un problema en cada propuesta.



Ejecutar en ordenador de ibm las 3 propeustas y apuntar los tiempos de ejecucion
Profunidad = numero de puertas basicas necesarias para mi circuito

Si la mas lenta es la que tiene menos qubits, argumentar que es la mejor porque el resultado obtenido es el mas fiable.

