\section{Basics of Quantum Systems}
\begin{introduction}
	\item Hadamard Operator
	\item One Qubit
	\item Quantum State
	\item Visualization of a (Real-Valued) Qubit
	\item Superposition and Measurement
\end{introduction}

\subsection{Hadamard Operator}
The quantum operator for the coin flipping of Fig.~(\ref{fig:coin-flipping}) is the Hadamard gate also called the H gate.
\begin{equation*}
	H=\begin{pmatrix}
			1 & 1 \\ 1 & -1
	\end{pmatrix}
\end{equation*}

The quantum operator for the mirror of Fig.~(\ref{fig:coin-flipping}) is the X gate.
\begin{equation*}
	X=\begin{pmatrix}
		  0 & 1 \\ 1 & 0
	\end{pmatrix}
\end{equation*}

The circuit of Fig.~(\ref{fig:coin-flipping}) can be expressed with quantum gates as:
\begin{equation*}
	HXH\ket{0}
\end{equation*}

The evolution of the ket is:
\begin{equation*}
	HXH\ket{0}=\frac{1}{\sqrt {2}}HX(\ket{0}+\ket{1})=\frac{1}{\sqrt {2}}H(\ket{1}+\ket{0})=\frac{1}{\sqrt {2}}\frac{1}{\sqrt {2}}(\ket{0}-\ket{1}+\ket{0}+\ket{1})=\frac{1}{2}(2\ket{0})=\ket{0}
\end{equation*}

\subsection{One Qubit}
But not only real numbers we used, complex numbers are also used in quantum computing.
When complex numbers are used, the state of a qubit can be represented by a four dimensional real number valued vector, which is not possible to visualize.
On the other hand, it is possible to represent such state in three dimensions (with real numbers).
This representation is called Bloch sphere.

In three dimensions, we have axes: x, y, and z.
X refers to the rotation with respect to x-axis.
Similarly, we have the rotation with respect to y-axis and z-axis.

In the literature, the quantum states obtained after applying $H$ to $\ket{0}$ and $\ket{1}$ are known as ket-plus $\ket{+}$ and ket-minus $\ket{-}$ states, respectively.

\subsection{Quantum State}
Remember:
\begin{itemize}
	\item A quantum state can be represented by a vector having length 1, and vice versa.
	\item Any length preserving (square) matrix is a quantum operator, and vice versa.
\end{itemize}

\subsection{Visualization of a (Real-Valued) Qubit}
We can draw a qubit if it has a real-valued coefficients over the computational base.

\subsection{Superposition and Measurement}

A quantum system can be in more than one state with nonzero amplitudes.

Then, we say that our system is in a superposition of these states.

When evolving from a superposition, the resulting transitions may affect each other constructively and destructively.

This happens because of having opposite sign transition amplitudes.

Otherwise, all nonzero transitions are added up to each other as in probabilistic systems.
Measurement

We can measure a quantum system, and then the system is observed in one of its states.
This is the most basic type of measurement in quantum computing.
(There are more generic measurement operators, but we will not mention about them.)

The probability of the system to be observed in a specified state is the square value of its amplitude.