\section{Basics of Classical Systems}
\begin{introduction}
	\item One Bit
	\item Coin Flip
	\item Probabilistic Operator
	\item Probabilistic Evolution
	\item Correlation
\end{introduction}

\subsection{One Bit}
Bit (or binary digit) is the basic unit of information used in computer science and information theory.
It can also be seen as the smallest `useful' memory unit, which has two states named 0 and 1.
At any moment, a bit can be in either state 0 or state 1.

We can apply four different operators to a single bit:
\begin{itemize}
	\item Identity: I(0)=0 and I(1)=1
	\item Negation: NOT(0)=1 and NOT(1)=0
	\item Constant (Zero): ZERO(0)=0 and ZERO(1)=0
	\item Constant (One): ONE(0)=1 and ONE(1)=1
\end{itemize}

Based on this observation, we can classify the operators into two types: Reversible and Irreversible.
\begin{itemize}
	\item If we can recover the initial value(s) from the final value(s), then the operator is called reversible like Identity and NOT operators.
	\item If we cannot know the initial value(s) from the final value(s), then the operator is called irreversible like ZERO and ONE operators.
\end{itemize}

\subsection{Coin Flip: A Probabilistic Bit}
We have two different outcomes: heads (0) and tails (1).
We use a column of size 2 to show the probabilities of getting heads and getting tails.
For the fair coin, our information after the coin-flip will be $\begin{pmatrix} 0.5\\0.5
\end{pmatrix}$.


In general, a probabilistic state is a vector with coefficients over the standard basis, with certain properties:
\begin{itemize}
	\item Each coefficient is non-negative.
	\item The summation of coefficients is 1.
\end{itemize}

\begin{definition}
We can say that a probabilistic state is a probability distribution over deterministic states.
We can show all information as a single mathematical object, which is called as a \textbf{stochastic vector}.
\end{definition}

\subsection{Probabilistic Operator}

\begin{definition}
The transition probabilities between two stochastics vectors is called a \textbf{probabilistic operator}, and can be represented as a square matrix called \textbf{stochastic matrix}.
\end{definition}

For a probabilistic matrix, each column represents the transition probabilities from a state to all states.
Therefore, the summation of all entries in each column is 1.

\subsection{Probabilistic Evolution}
A probabilistic state is a stochastic vector, say $v$.
A probabilistic operator is a stochastic matrix, say $A$,
If probabilistic operator A is applied to probabilistic state $v$, the new state, say $v^\prime$, is calculated as
\begin{equation*}
v^\prime=Av
\end{equation*}
\begin{definition}
In this situation, we call that $v^\prime$ is a \textbf{probabilistic evolution} of $v$.
\end{definition}

\subsection{Correlation}
If the state of a composite system cannot be written as the tensor product of the states of its sub-systems, then we can say that the sub-systems are correlated.