\section{Quantum Operators on a (real-valued) Qubit}
\begin{introduction}
	\item Operations on the Unit Circle
	\item Rotations
	\item Reflections
	\item Quantum Tomography
\end{introduction}

\subsection{Operations on the Unit Circle}
Any real-valued quantum state $\ket{a}$ is a point in the unit circle, and it can be described by an angle, say $\theta\in [0, 2\pi)$:
\begin{equation*}
	\ket{a} = (\cos\theta, \sin\theta)
\end{equation*}

We can set the qubit to the state $\ket{a}$ by using a rotation operator between $\ket{0}$ and $\ket{1}$ with angle $\theta$.

\subsection{Rotations}
The matrix form of a rotation is as follows:
\begin{equation*}
	R(\theta)=\begin{pmatrix}
		          \cos\theta & \sin\theta\\ -\sin\theta & \cos\theta
	\end{pmatrix}
\end{equation*}

where $\theta$ is the angle of rotation (in counter-clockwise direction).

\subsection{Reflections}
The following operators are reflections on the unit circle.
\begin{itemize}
	\item Z: The line of reflection is x-axis.
	\item X: The line of reflection is y=x.
	\item H: The line of reflection is y=$\tan(\pi/8)$x.
\end{itemize}

The matrix form of reflection with the angle $\theta$ of the line of reflection, is represented as follows:
\begin{equation*}
\gate{Ref}(\theta)=\begin{pmatrix}
	                   \cos(2\theta) & \sin(2\theta) \\ \sin(2\theta) & -\cos(2\theta)
\end{pmatrix}
\end{equation*}

\subsection{Quantum Tomography}
We study a simplified version of quantum tomography here.
Suppose that you are given 1000 copies of a qubit and your task is to learn the state of this qubit.

But this is not possible, because if we have a qubit and its state is either $\ket{a}$ or $-\ket{a}$, after measurement, the probabilities of observing state $\ket{0}$ and state $\ket{1}$ are the same for $\ket{a}$ and $-\ket{a}$.
Therefore, we cannot distinguish them.

Even though the states $\ket{a}$ and $-\ket{a}$ are different mathematically, they are assumed as identical from the physical point of view.