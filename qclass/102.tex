\section{Introduction to Quantum Mechanics}
\begin{introduction}
	\item Dirac Notation (bra-ket notation)
	\item Hilbert Space
	\item Qubits
	\item Postulates of Quantum Mechanics and its Application to Qubits
\end{introduction}

\subsection{Dirac Notation (bra-ket notation)}

Dirac notation or bra-ket notation, is used in linear algebra on complex vector spaces along with their dual space in both the finite-dimensional and infinite-dimensional cases.
It is specifically designed to facilitate the types of calculations that frequently arise in quantum mechanics.
Bra-ket notation was created by Paul Dirac in 1939.

In mathematics and physics textbooks, vectors are often distinguished from scalars by writing an arrow over the identifying symbol.
In the Dirac notation, the symbol identifying a vector is written inside a ket, $\ket{a}$, the dual vector with a bra, $\bra{a}$ and the inner products will be written as bra-ket $\braket{a}{b}$.

If $\bra{a}$ is a normalized vector, the operator $\hat{P}$, called projection operator, $\hat{P}=\ketbra{a}{a}$ picks out the portion of any other vector that lies along $\ket{a}$:
\begin{equation*}
	\hat{P}\ket{b}=\ketbra{a}{a}\ket{b}=\braket{a}{b}\ket{a}
\end{equation*}

\subsection{Hilbert Space}

A Hilbert space $\H$ is a vector space with an inner product.

The Hilbert spaces that we are going to consider here will be those that go over the field of complex numbers with  finite dimension $2^n$, for some positive integer n.
We can choose a basis and alternatively represent vectors in this basis as finite column vectors, and represent operators with finite matrices as well.


\subsection{Qubit}

A qubit is the basic unit of quantum information (the quantum version of the classic binary bit).
A qubit is a two-state (or two-level) quantum mechanical system; therefore, qubits live within the Hilbert space of dimension 2.

\subsection{Postulates of Quantum Mechanics, and its Application to Qubits}
Ver TFM.