\section{Introduction}

Quantum circuits, pioneered by Deutsch in~\cite{Deutsch1989}, are widely acknowledged as a practical and commonly used approach for illustrating the operations of quantum gates on qubits, ultimately describing the unitary matrix of a quantum computer through graphical representations.
Consequently, quantum algorithms and protocols are typically depicted in the format of quantum circuits.
This leads to a challenge when dealing with complex quantum algorithms or protocols, as their circuit graphs can quickly surpass manageable sizes, making drawing and manipulation impractical.

In classical computing, circuits can be not only graphically represented but also expressed as Boolean expressions using Boolean gates that implement logic based on truth tables.
These Boolean expressions are well-suited for algebraic manipulations.
Conversely, the derivation of an algebraic expression for a quantum circuit has been challenging because quantum computing lacks a language analogous to Boolean expressions in classical computing.
The present paper aims to introduce a formal semantics for a novel algebraic language designed to streamline the expression of quantum circuits in a concise manner so that fundamental algebraic laws for quantum circuits can be easily proven, and complex quantum algorithms can be simplified when expressed in this language.

A bibliographic review shows the great interest that the problem arouses, although no solution has been fully accepted to date.

Note that the present work does not require the use of Mathematica or matrices.

Another close paper is~\cite{Ying}, which presents the design of an algebraic language for formally specifying quantum circuits in distributed quantum computing so that using that language, quantum circuits can be represented in a compact way.
However, the language proposed in the present work is simpler and closer to simulation programming, which facilitates its implementation.
Indeed, this issue is demonstrated through the presentation of a first version of a software implementation of the proposal.
In particular, the present notation proposal is more intuitive as it sets the quantum gates in order of execution (from left to right).

The equational theory of quantum circuits is discussed in the paper~\cite{Staton}, which presents an axiomatization of the relationship between measurement, qubit initialization and a set of unitary gates.
Another recent work~\cite{Wang19} proposes the unification of quantum and classical computing in open quantum systems.
Also recently, the paper~\cite{doi:10.1098/rsta.2019.0161} describes the flow of quantum compilation with several NP-hard tasks and proposes algorithms based on Boolean satisfiability to address those computationally complex problems.
Note that the three aforementioned recent papers address related but different topics than the one addressed in the present paper.

Finally, one of the latest related works is~\cite{Wang21} on the equivalence of dynamic quantum circuits.
That proposal looks promising, but has only been evaluated on toy examples, with no available software packages for checking yet.