\section{Introduction}

Quantum circuits, pioneered by Deutsch in~\cite{Deutsch1989}, are widely acknowledged as a practical and commonly used approach for illustrating the operations of quantum gates on qubits, ultimately describing the unitary matrix of a quantum computer through graphical representations.
Consequently, quantum algorithms and protocols are typically depicted in the format of quantum circuits.
This leads to a challenge when dealing with complex quantum algorithms or protocols, as their circuit graphs can quickly surpass manageable sizes, making drawing and manipulation impractical.

In classical computing, circuits can be not only graphically represented but also expressed as Boolean expressions using Boolean gates that implement logic based on truth tables.
These Boolean expressions are well-suited for algebraic manipulations.
Conversely, the derivation of an algebraic expression for a quantum circuit has been challenging because quantum computing lacks a language analogous to Boolean expressions in classical computing.

In~\cite{reina2024ix} Jorge Garcia-Diaz et al. presents a formal semantics for a novel algebraic language designed to
streamline the expression of quantum circuits in a concise manner so that fundamental algebraic laws for quantum circuits can be easily proven, and complex quantum algorithms can be simplified when expressed in this language.

Now in this work we present an extension to the notation that allows us to more easily handle an arbitrarily large number of indices, for this we introduce the notation and the basic operations when the indication of the indices is based on sets.