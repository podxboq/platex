\subsection{Qubits}
As introduced in~\cite{diracBraket}, qubits are normally written in the so-called bra–ket notation.
Each qubit is represented as a ket, of the form $\ket{a}$, which denotes a vector in $\Q$.

For instance, the computational basis states are written  as $\ket{0}$ and $\ket{1}$, and called ket 0 and ket 1, respectively (see~\cite{NielsenChuang}).
Thus, any qubit can be described by a linear combination of these two basis states as $\ket{a} =\alpha_0\ket{0} +\alpha_1 \ket{1}$, where $\alpha_0,  \alpha_1 \in \mathbb{C}$ such that $ |\alpha_0|^{2}+|\alpha_1|^{2}=1$.

\begin{notation}
	For any qubit $\ket{a}$, the following notation is used:
	\[a_{0}=\braket{a}{0}\text{ and }a_{1}=\braket{a}{1},\]
\end{notation}
so the qubit $\ket{a}$ can be write by:
\begin{equation}
	\label{eq:qubit-coordenadas}
	\ket{a}=a_0\ket{0}+a_1\ket{1}
\end{equation}

For other basis, like for example $\set{\ket{+}, \ket{-}}$, where $\ket{+}=\frac{\ket{0}+\ket{1}}{2}$ and $\ket{-}=\frac{\ket{0}-\ket{1}}{2}$, then a qubit $\ket{a}$ can be described like:
\begin{equation*}
	\ket{a}=a_+\ket{+}+a_-\ket{-}
\end{equation*}