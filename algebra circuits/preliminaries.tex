\section{Preliminaries}
The fundamental information unit within a quantum computing system is the qubit, which is denoted by a unitary vector within a Hilbert space of dimension 2 denoted by $\Q$.
Multiple qubits can be taken together so that an element of the Hilbert space of $n$ qubits is considered and denoted by $\Q_n$.
This case can be seen as the tensor product of the $n$ Hilbert spaces, $\Q_n=\Q^{\otimes n}=\Q\otimes\dots\otimes\Q$ corresponding to each qubit in the system, so that the resulting space dimension is $2^n$.

Let $\N$, $\R$ and $\C$ denote the sets of natural, real and complex numbers, respectively.
Since the $0$ is not a natural number, let $\N^*$ be the natural number within cero.
The notation $\N_n$ and $\N^*_n$ is used to denote the subset of $\N$ and $\N^*$ respectively that contains the natural numbers smaller or equal than $n$.
In set theory, for any finite set $S$, its number of elements  $|S|$ is called the cardinality of $S$.

Let $\B$ denote the set $\set{0,1}$ and the Kronecker delta is the function $\delta:\N^*\times\N^*\longrightarrow\B$ defined by $\delta(k,l)=\delta_{k,l}=1$ if $k=l$ and $0$ otherwise.
For a set $\Gamma\subseteq\N^*_n$ the value of $\delta_{k,\Gamma}$ is $1$ if $k\in\Gamma$ and $0$ otherwise.