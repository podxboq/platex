\begin{abstract}
		The use of graphical language in quantum computing for the representation of algorithms, although intuitive, is not very useful for different tasks such as the description of quantum circuits in text environments, the calculation of quantum states or the optimization of quantum circuits.
		While classical circuits can be represented either by circuit graphs or by Boolean expressions, quantum circuits have until now predominantly been illustrated as circuit graphs because no formal language for quantum circuits that allows algebraic manipulations has so far been accepted.
		There is a proposal to represent quantum circuits in a convenient and concise manner,
		similar to the way Boolean expressions are used in classical circuits.
		The proposed notation allows the consistent and parameterized description of quantum algorithms, as well as the easy handling of the elements that compose it to achieve powerful optimizations in the number of gates of the circuits.
		Based on this proposal, we made an extension to the notation that allows us to more easily handle an arbitrarily large number of indices, for this we introduce the notation and the basic operations when the indication of the indices is based on sets.
\end{abstract}
